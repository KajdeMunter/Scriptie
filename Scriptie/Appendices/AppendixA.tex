% Appendix A

\chapter{Code} % Main appendix title

\label{BijlageCode} 

\section{Docker-compose opstelling voor k6, InfluxDB \& Grafana}

\label{Bijlagek6}

Om de loadtest met k6, influxDB en grafana op te stellen heeft Loadimpact een docker-compose opstelling gemaakt. Na wat onderzoek is het opgevallen dat deze opstelling erg verouderd is. Daarom is ervoor gekozen om een eigen opstelling te maken:
\begin{minted}{yaml}
version: '3.4'

networks:
  k6:
  grafana:

services:
  influxdb:
    image: influxdb:1.5.4
    networks:
    - k6
    - grafana
    ports:
      - "8086:8086"
    environment:
      - INFLUXDB_DB=k6
    
  grafana:
    image: grafana/grafana:6.4.1
    networks:
      - grafana
    ports:
      - "3000:3000"
    environment:
      - GF_AUTH_ANONYMOUS_ORG_ROLE=Admin
      - GF_AUTH_ANONYMOUS_ENABLED=true
      - GF_AUTH_BASIC_ENABLED=false
    volumes:
      - ./grafana/datasource.yml:/etc/grafana/provisioning/datasources/datasource.yml
  
  k6:
    image: loadimpact/k6:0.25.1
    networks:
      - k6
    ports:
      - "6565:6565"
    environment:
      - K6_OUT=influxdb=http://influxdb:8086/k6
    volumes:
      - ../k6:/k6
\end{minted}

Hiervoor is een Pull-request gedaan naar loadimpact/k6 om dit te verbeteren. \texttt{https://github.com/loadimpact/k6/pull/1183} samen met de issue\\ \texttt{https://github.com/loadimpact/k6/issues/1182}. Hierin is te lezen wat precies de veranderingen waren. De loadtest is geschreven in javascript met de volgende code:
\begin{minted}{javascript}
import http from "k6/http";
import { sleep, check } from "k6";

export let options = {
  stages: [
    { duration: "10s", target: 20 },
    { duration: "10s", target: 40 },
    { duration: "10s", target: 60 },
  ]
};

export default function() {
  check(http.get("https://test.developers.nl/"), {
    "is status 200": (r) => r.status === 200
  });
  sleep(1);
};
\end{minted}