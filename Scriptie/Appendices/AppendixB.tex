% Appendix B

\chapter{Tabellen} % Main appendix title

\label{BijlageTabellen} 

\section{Beyond the 12-factor app}

\label{TabelFactors}

In deze tabel worden de 15 factoren behandeld.

\begin{longtable}[c]{c p{3cm} p{3.5cm} p{5.5cm}}
	\toprule
	\textbf{Factor} & \textbf{Naam} & \textbf{Gevolg} & \textbf{Waarom?} \\
	\midrule		
	1 & One codebase, one application & Onderhoudbaarheid & Door de frequente deploys kunnen veranderingen snel in productie gezet worden. \\
	2 & API first & Structural scalability & Door de API op de eerste rang te zetten van het development proces wordt de mogelijkheid gecreëerd om met elkaars contracten te communiceren zonder interne ontwikkelingsprocessen te verstoren. Zo kunnen veel nieuwe services gemakkelijker worden toegevoegd. \\
	3 & Dependency management & Onderhoudbaarheid & Gemakkelijk opzetten van project voor nieuwe ontwikkelaars. \\
	4 & Design, build, release, and run & Onderhoudbaarheid & Het is onmogelijk om veranderingen aan de code tijdens runtime te maken. \\
	5 & Configuration, credentials, and code & Structural scalability & Environment variabelen zijn niet in omgevingen maar per deployment opgezet, zo maakt de hoeveelheid omgevingen niet uit. \\
	6 & Logs & Onderhoudbaarheid & Door logs naar de \texttt{stdout} te sturen is het gemakkelijker om specifieke fouten te vinden, overzicht te creëren en actief meldingen te versturen naar ontwikkelaars. \\
	7 & Disposability & Load scalability & Door processen gemakkelijk te laten stoppen en starten gaat het schalen een stuk sneller. \\
	8 & Backing services & Onderhoudbaarheid & Door backing services als \enquote{attached resources} te behandelen maakt het niet uit welke techniek er wordt gebruikt en zijn deze dus los gekoppeld. \\
	9 & Environment parity & Onderhoudbaarheid & Hierdoor kan een stuk vaker gedeployed worden. \\
	10 & Administrative processes & Onderhoudbaarheid & Door commands in versiebeheer op te slaan is er een duidelijk overzicht en een geschiedenis van alle \enquote{one-off processes} die gebeuren. \\
	11 & Port binding & \textbf{Begrijp ik nog niet} & \textbf{Vaag} \\
	12 & Stateless processes & Load scalability & Mede door de shared-nothing architectuur kan het systeem gemakkelijker schalen. \\
	13 & Concurrency & Load scalability & Door processen gemakkelijk te laten stoppen en starten gaat het schalen een stuk sneller. \\
	14 & Telemetry & \textbf{kaas} & \textbf{kaas} \\
	15 & Authentication and authorization & \textbf{kaas} & \textbf{kaas} \\
	\bottomrule\\
\end{longtable}

\label{Bijlagek6}