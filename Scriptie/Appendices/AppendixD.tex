\chapter{Feedback van bedrijfsbegeleider}

\label{FeedbackVanBedrijfsbegeleider} 

\subsubsection{Professional skills}
\textit{Aandachtspunten hierbij zijn: communicatie en samenwerking}.

\underline{Sterke punten}\\[0.1cm]
Kaj laat goed weten waar hij mee bezig is en geeft tijdig aan als hij ergens niet uitkomt. Qua samenwerking is kaj top! Hij is nooit te beroerd iets op te pakken als het hem wordt gevraagd. Alle collega’s zijn altijd erg blij als Kaj aanwezig is en hij is een gewaardeerde collega.

\underline{Verbeterpunten}\\[0.1cm]
Tijdig communiceren over organisatorische zaken. Dit heeft ook wel met de communicatie binnen school te maken helaas maar het was soms op het laatste moment dat iets veranderd werd.


\subsubsection{Manage \& Control}
\textit{Aandachtspunten hierbij zijn: planmatig werken, projectvoortgang bewaken, principes toepassen om een softwareontwikkelproces te managen en te bewaken.}

\underline{Sterke punten}\\[0.1cm]
Kaj werkt volgens de juiste principes, hij heeft een goede kwalitatieve werkwijze. Hij heeft een duidelijke aanpak voor wat betreft zijn project en het is voor iedereen duidelijk geweest wanneer hij waar aan heeft gewerkt.

\underline{Verbeterpunten}\\[0.1cm]
Sneller om duidelijkheid vragen bij de opleiding.


\subsubsection{Analyseren}
\textit{Aandachtspunten hierbij zijn: doelgerichtheid, keuze van de juiste deelvragen die beantwoord moeten worden om tot een goed beroepstechnische probleemstelling  te komen, diepgang van de analyses, kritisch gebruik bronnen, helder vaststellen van de problemen.}

\underline{Sterke punten}\\[0.1cm]
Kaj is in het gehele traject zeer sterk geweest in het achterhalen van requirements, met name door een goede analyse van de huidige en gewenste toekomstige situatie. Hij stelt kritische vragen met een goed onderbouwde argumenten waarbij ook de nodige security gerelateerde onderwerpen zijn besproken. Kaj heeft zelf ook actie ondernomen door het maken van pull-requests en tickets bij o.a Ansible en Bitbucket en ook het schrijven van een blog om een stelling te kunnen toetsen.

\underline{Verbeterpunten}\\[0.1cm]
Bij het analyseren van problemen wil Kaj nog wel eens te veel in een keer proberen op te lossen. Het opdelen van issues in kleinere stukken maakt het overzichtelijker waar het probleem vandaan komt.



\subsubsection{Adviseren}
\textit{Aandachtspunten hierbij zijn: gemaakte keuzes en onderbouwing hiervan, volledigheid (bijvoorbeeld kwaliteitsaspecten, security, schaalbaarheid, performance, privacy), presentatie van de adviezen.}


\underline{Sterke punten}\\[0.1cm]
Kaj is goed op de hoogte van de huidige stand van technische ontwikkelingen op DevOps gebied. In zijn scriptie geeft hij een volledig beeld van de te nemen stappen voor de volgende iteratie van onze Continuous Integration / Continuous Delivery (CI/CD) pipeline. De onderbouwing van de gemaakte keuzes en adviezen is goed en omvat alle bovenstaande aspecten.

\underline{Verbeterpunten}\\[0.1cm]
In het advies blijft het kostenaspect van de te gebruiken oplossingen enigszins onderbelicht. 


\subsubsection{Ontwerpen}
\textit{Aandachtspunten hierbij zijn: keuze van relevante ontwerptechnieken (FO, TO, UI, DB), kwaliteit van de ontwerpen, architectuur, teststrategie, toetsing van de ontwerpen (prototyping, voorleggen aan experts).}

\underline{Sterke punten}\\[0.1cm]
Kaj heeft veel werk verzet met het ontwerpen, testen en uitwerken van de CI/CD pipeline architectuur. Met name het samenbrengen van de verschillende door Developers.nl beheerde applicaties in een werkwijze is waardevol. 

\underline{Verbeterpunten}\\[0.1cm]
Bij het uitwerken van de architectuur eerder toetsen of deze voldoende is uitgewerkt.


\subsubsection{Realiseren}
\textit{Aandachtspunten hierbij zijn: kwaliteit van het beroepstechnische  en kwaliteit van het ontwikkelproces. Clean code, documenteren, testen, integreren, continious integration \& deployment, gebruik frameworks.}

\underline{Sterke punten}\\[0.1cm]
Met de huidige applicaties en tooling als basis heeft Kaj de nodige aanpassingen (laten) doen aan de bestaande applicatie architecturen om deze vervolgens met de door hem ontworpen CI/CD straat te kunnen deployen. Een ander heeft hij ook al reeds door middel van een Proof of Value geïntroduceerd. Op dit moment ondergaat deze PoV de nodige tests en we verwachten dit op korte termijn in productie te gaan nemen. Daarnaast heeft hij ook het vervolgtraject beschreven waarmee wij de volgende stap kunnen gaan zetten.

\underline{Verbeterpunten}\\[0.1cm]
Op het gebied van monitoring is nog geen centraal dashboard beschikbaar.


\subsubsection{Aanvullende observaties}
Ik heb Kaj ervaren als een prettige collega en sparringpartner. Een echte professional met een prettige instelling en een goed kennisniveau. 
