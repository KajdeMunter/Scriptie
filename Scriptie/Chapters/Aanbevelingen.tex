\chapter{Aanbevelingen}

\label{Chapter8}

\section{Cloud service providers}
Momenteel worden de applicaties binnen Developers.nl gehost op een simpele, traditionele server van TransIP. Een overweging om te maken is of dit niet beter naar een cloud service provider kan worden verplaatst, aangezien dit mogelijk de onderhoudbaarheidslast verminderd. L. Wang \textit{et al.} \parencite{CloudPerspective} definiëren cloud computing als \enquote{A computing Cloud is a set of network enabled services, providing scalable, QoS guaranteed, normally personalized, inexpensive computing infrastructures on demand, which could be accessed in a simple and pervasive way}. Volgens \parencite{CloudPlatformsIntroduction} zijn er drie verschillende categorieën van Cloud computing:
\begin{itemize}
	\item Infrastructure as a Service (IaaS): Een virtueel aangeboden infrastructuur van rekenkracht en/of geheugen \parencite{CloudComputingAdvantages}.
	\item Platform as a Service (PaaS): Een aangeboden platform voor ondersteuning van deployment, ontwikkelen en testen van applicaties \parencite{TransformingCloud}.
	\item Software as a Service (SaaS): Een aangeboden (web)applicatie dat direct gebruikt kan worden \parencite{CloudComputingAdvantages}.
\end{itemize}

Om te overwegen of een cloud provider bij de wensen van Developers.nl past worden de voor-en nadelen op een rijtje gezet:
\subsubsection{Voordelen Cloud}
De kosten van cloud hosting zijn flexibel, er wordt alleen betaalt voor wat er daadwerkelijk gebruikt wordt, zolang er maar verstandig gebruik van wordt gemaakt. Dit betekend dat het mogelijk erg kostenefficiënt kan zijn voor Developers.nl aangezien er tijdens de maandelijkse \enquote{Tech Nights} piekmomenten zijn op de website, en er heel weinig verkeer is op het EMS. Een ander voordeel is dat er bijna een ongelimiteerde hoeveelheid opslagruimte beschikbaar is. Aangezien het CMS van de website dubbel functioneert als \enquote{file-server} en er veel foto's worden geüpload is het fijn dat er geen rekening hoeft worden gehouden met de hoeveelheid opslag. Bovendien worden software updates automatisch uitgevoerd, software als K8s kunnen al inbegrepen zijn bij de infrastructuur en het maakt schalen gemakkelijker. Aangezien cloud providers meer middelen voor beveiliging hebben wordt de veiligheid ook een stuk verbeterd.

\subsubsection{Voordelen traditioneel} 
Ook al is cloud hosting meer kostenefficiënt is het toch mogelijk dat een web host goedkoper uitkomt. Zolang er maar geen hoge piekmomenten zijn in het verkeer. Dit is dus niet van toepassing op Developers.nl aangezien de \enquote{Tech Nights} of andere evenementen voor piekmomenten zorgen. 

\subsubsection{Nadelen cloud}
Cloud providers hebben de mogelijkheid voor technische problemen buiten de controle van klanten, waardoor het mogelijk is dat er downtime ontstaat. Ook kan het duurder uitpakken zodra er niet goed word omgegaan met de schaalstrategie of benodigde rekenkracht.

\subsubsection{Nadelen traditioneel}
De mogelijkheid bestaat dat er meer kosten worden gemaakt dan nodig is. Ook is shared-hosting een risico omdat zodra een andere klant veel rekenkracht opeist de kans ontstaat dat de prestaties dalen.

\subsubsection{Aanbeveling}
Er zijn veel verschillende cloud providers, waarvan de grootste Amazon Web Services (AWS), Microsoft Azure en Google Cloud Platform (GCP) zijn. Er moet onderzoek worden uitgevoerd over de kosten van het overstappen. Het is verstandig om samen met een migratie naar een cloud provider ook zaken mee te nemen als e-mail beheer in het bedrijf, of de inbegrepen software pakketen als Microsoft Office of Google Drive.

\section{Serverless computing}
In verband met de lage hoeveelheid verkeer op het EMS is \enquote{serverless computing} een goede oplossing om kosten te besparen.

\section{Container Orchestration}
Container Orchestration met Docker Swarm zou het mogelijk maken om de website op te schalen. Er is bewezen dat de website schaalbaar is, dus er zijn geen implicaties voorzien bij het implementeren. Wel moet er rekening gehouden worden met Traefik, en de integratie hiervan met Docker Swarm.

\section{Één generieke infrastructuur}
Het maken van een enkele ansible infrastructuur zou het concept \enquote{one codebase, one application} van de 15-factor app verbeteren.