\chapter{Conclusie}

\label{Chapter9}

In dit hoofdstuk wordt antwoord gegeven op de hoofdvraag: \enquote{\hoofdvraagname}.

\section{Verwachtingen}
Developers.nl wilt onderhoudbaarheid bereiken door kwaliteitsstandaarden af te dwingen. Ook wilt Developers.nl twee soorten schaalbaarheid. Eén in de vorm van het deployen van verschillende branches in aparte omgevingen, en één in de vorm van horizontaal schalen om meer verkeer aan te kunnen. Deze twee soorten kunnen niet gerealiseerd worden zonder de onderhoudbaarheid te waarborgen. Er zijn vijf must-have requirements:

\begin{itemize}
	\item De oplossing moet méér dan twee unieke instanties van de website naast elkaar kunnen draaien.
	\item De oplossing moet moet unieke instanties van de website automatisch kunnen aanmaken.
	\item De oplossing moet een methode bevatten om kwaliteit van nieuwe toevoegingen aan de infrastructuur automatisch te waarborgen.
	\item Minimaal één monitoring tool voor het monitoren van performance.
	\item De oplossing moet voldoen aan één of meerdere kwaliteitsstandaarden.
\end{itemize}

\section{Technieken}
De 12-Factor App is een methodologie die twaalf best-practices samenvoegt om moderne, schaal- en onderhoudbare web-applicaties te bouwen. Het boek \enquote{Beyond the 12-factor app} \parencite{Beyond12Factor} is hierop verder gegaan door nog een drietal factoren toe te voegen. Door een applicatie te evalueren op deze vijftien factoren, samen met de definitie van ISO-25010 \parencite{ISO25010} is te beoordelen of deze schaal- en onderhoudbaar is. Om de schaalbaarheid van een systeem te waarborgen zijn  de methoden van Weinstock en Goodenough \parencite{OnSystemScalability} een geschikte manier.

\section{Huidige situatie}
Om de deelvraag \enquote{\deelhuidig} te beantwoorden zijn door dit onderzoek meerdere punten van verbetering gevonden. Een verbeterpunt in de functionele schaalbaarheid is de hoeveelheid opslag van de server. Deze kan snel vol raken door ongebruikte Docker volumes en images die ontstaan bij een deployment. 

Ook wordt factor 1 (One codebase, one application) van de 15-Factor App niet volledig opgevolgd. De infrastructuur wordt op meerdere plekken opgebouwd en zou netter staan in een aparte codebase. De applicaties voldoen aan factor 13 (Concurrency) maar er wordt nog geen gebruik van gemaakt. 

Factor 11 (Port binding) is voor PHP niet verstandig aangezien PHP juist ontworpen is om een webserver ervoor te hebben, dit verslechtert dus de onderhoudbaarheid. Er zijn voor zowel het EMS als de website geen monitoring tools in gebruik voor performance, dat betekend dat factor 14 (Telemetry) beter kan. Er is geen concrete manier om tests uit te voeren of aan testcriteria is voldaan. Hierdoor is de Testbaarheid van de systemen minimaal.

\section{Verbeteringen}
Om de kenmerken modulariteit en herbruikbaarheid van ISO 25010 te verbeteren kan er een centrale infrastructuur IaC repository gemaakt worden met Ansible. Om de kenmerken analyseerbaarheid, testbaarheid en wijzigbaarheid te verbeteren kunnen policies worden afgedwongen door middel van PaC. Voor het waarborgen van testbaarheid is Codecov een geschikte tool. Om onderhoudbaarheid te verbeteren en schaalbaarheid te bewijzen is het concept van feature-environments erg geschikt. Developers.nl ziet dit concept graag in de praktijk, dit heeft dan ook de hoogste prioriteit als uitkomst van dit onderzoek.

\section{Implementatie}
Er zijn vijf verbeteringen geïmplementeerd: Feature-environments, Codecov, Opschonen van Docker images, Open Policy Agent, en Grafana.

Codecov is in de back-end van de website geïmplementeerd om coverage te visualiseren op Pull-Requests, het genereren hiervan gebeurt in de \texttt{php7-fpm} Dockerfile in een test-buildstage.

Docker images worden opgeschoond door middel van de Ansible \texttt{docker\_prune} module. Deze ruimt images ouder dan 4 uur op bij elke deployment.

De implementatie van Feature-environments heeft de basis gelegd voor een aantal opvolgende implementaties. Het heeft de infrastructuur, pipeline, en workflow veranderd door Traefik -- een reverse proxy -- te implementeren die er voor zorgt dat meerdere individuele omgevingen naast elkaar kunnen draaien, die te bereiken zijn door subdomeinen. Ook beveiligt het de Docker Socket door de TCP connectie met TLS te encrypten. 

De implementatie van Policy-as-Code gaat verder op de beveiliging van de Docker Socket in en gebruikt Open Policy Agent om de users van docker commando's te limiteren. Ook staat er een opzet voor het evalueren van build-time policies.

Verder wordt Traefik gebruikt om Grafana -- een monitoring dashboard -- te exposen op een subdomein.


\section{Requirements}
Must-requirements:
\begin{itemize}
	\item De oplossing moet méér dan twee unieke instanties van de website naast elkaar kunnen draaien.
    \begin{itemize}
        \item Behaald.
    \end{itemize}

	\item De oplossing moet moet unieke instanties van de website automatisch kunnen aanmaken.
    \begin{itemize}
        \item Behaald tot op zekere hoogte, de instanties kunnen nog niet automatisch worden verwijderd. Hier is een aanbeveling over gemaakt.
    \end{itemize}

	\item De oplossing moet een methode bevatten om kwaliteit van nieuwe toevoegingen aan de infrastructuur automatisch te waarborgen.
    \begin{itemize}
        \item Behaald. Er is een aanbeveling gemaakt om hier verder op in te gaan.
    \end{itemize}

	\item Minimaal één monitoring tool voor het monitoren van performance.
    \begin{itemize}
        \item Niet volledig behaald. Grafana is geïmplementeerd om data te visualiseren, maar er is nog geen data. 
    \end{itemize}

	\item De oplossing moet voldoen aan één of meerdere kwaliteitsstandaarden.
    \begin{itemize}
        \item Behaald, maar er zijn nog verbeteringen. Hier is een aanbeveling over gemaakt.
    \end{itemize}
\end{itemize}

Over de Should, Could en Won't requirements zijn aanbevelingen gemaakt in hoofdstuk \ref{Chapter8}.

\section{Hoofdvraag}

Developers.nl kan schaal- en onderhoudbaarheid verbeteren door middel van de volgende technieken: 
\begin{itemize}
    \item Feature-environments
    \item Codecov
    \item opschonen van Docker images
    \item Open Policy Agent
    \item Grafana en Prometheus
    \item AWS of Azure
    \item Docker Swarm
    \item Eén generieke infrastructuur repository
\end{itemize}
