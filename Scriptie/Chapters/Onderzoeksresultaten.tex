\chapter{Onderzoeksresultaten}

\label{Chapter3}

Dit hoofdstuk heeft als doel antwoord te geven op de onderzoeksvraag: \enquote{\hoofdvraagname}.

\section{Huidige situatie}
Deze paragraaf gaat over de deelvraag \enquote{\deelhuidig}

\subsection{Huidige Architectuur}
De huidige website is een combinatie van een PHP \& Symfony back-end API en Content Management Systeem, samen met een React + next.js front-end. De infrastructuur is momenteel gebouwd op Docker(-compose) + Ansible. Bitbucket pipeline wordt gebruikt voor het Continuous Integration / Deployment. 

In figuur \ref{fig:infra} is een component diagram te vinden van de huidige website structuur. De front-en backend structuur bevat 5 docker containers:
\begin{itemize}
	\item \textbf{PHP-FPM} (back-end)
	\item \textbf{Nginx} (front-en backend)
	\item \textbf{Redis} (back-end)
	\item \textbf{NodeJS} (front-end)
	\item \textbf{PostgreSQL} (back-end)
\end{itemize}

\begin{figure}
	\centering
	\includegraphics[width=13cm]{Figures/Infrastructure}
	\decoRule
	\caption[Infrastructuur]{Infrastructuur website front-en backend \textbf{HOE CITEREN?}}
	\label{fig:infra}
\end{figure}

PHP-FPM is een FastCGI Process Manager, deze Container serveert de Symfony “FosREST” API en het Content Management Systeem. De NodeJS container serveert een statische Next.js React applicatie en maakt gebruik van Server Side Rendering. Er zit een Nginx reverse proxy in die kiest om een request naar het back-end of de front-end te laten gaan. Redis is een Key-Value Database dat gebruikt wordt voor het cachen, en een PostgreSQL container als database. De Bitbucket Pipeline gebruikt Ansible om op de servers de geüpdatete containers te pullen en te starten.

Voor zowel de front- als backend is één monitoring tool genaamd \enquote{Sentry} geimplementeerd. Sentry creëert een duidelijk overzicht voor alle errors die opkomen in productie.

Ook heeft Developers.nl een \enquote{Employee Management Systeem} (EMS) gebouwd. Deze heeft een soortgelijke structuur aan de website. Het EMS bevat zeer veel gevoelige informatie en het is dus van hoog belang dat deze goed beveiligd is.

\subsection{Metingen}

Nu luidt de vraag, hoe schaalbaar is deze infrastructuur eigenlijk? Om dit te beantwoorden worden de verschillende definities van schaalbaarheid individueel behandeld. 

\subsubsection{Structural scalability}
Definitie: Het vermogen van een systeem om uit te breiden in een gekozen dimensie zonder ingrijpende wijzigingen in de architectuur.

Bij structural scalability hoort factor 3 (Config) van de 12-factor app. Een test om te bewijzen dat alle configuratie correct uit de code is verwerkt, is of de applicatie op elk gewenst moment open-source kan worden gemaakt zonder geclassificeerde informatie vrij te geven.

Voor de website wordt er gebruik gemaakt van docker-secrets en ansible-vault. Deze combinatie zorgt ervoor dat er nooit wachtwoorden, API sleutels en dergelijke plain-text in versiebeheer komt te staan. Deze secrets worden uiteindelijk in de containers als environment variabelen opgeslagen en uitgelezen door Symfony. In het EMS is deze techniek nog niet gebruikt en staan credentials wél plaintext in de repository.

Om aan factor 3 van de 12-factor app te voldoen moeten de configuratiefiles niet per specifieke omgeving (productie, test, staging) gegroepeerd worden maar moeten juist individueel per deployment geregeld worden. Dit gebeurt in zowel het EMS als de website, de bitbucket pipeline heeft zijn eigen specifieke environment variabelen om te gebruiken en de variabelen in de docker containers worden meegegeven in de algemene docker-compose file die in elke deployment hetzelfde zal zijn.

\subsubsection{Load scalability}
Definitie: Het vermogen van een systeem om elegant te presteren naarmate het aangeboden verkeer toeneemt.

Meerdere tools vergeleken:
\begin{itemize}
	\item https://loader.io/ (gratis versie te weinig)
	\item https://gatling.io/ (Is in Ruby)
	\item https://k6.io/
	\item http://tsung.erlang-projects.org/ (XML, Erlang)
\end{itemize}

Gekozen voor K6 omdat zo goed als elke ontwikkelaar genoeg Javascript kennis heeft om deze tool te gebruiken. Ook heeft k6 een gemakkelijke manier om de hoeveelheid Virtual Users (VU) geleidelijk te verhogen. Om de uitkomsten te visualiseren is InfluxDB samen met Grafana gebruikt. In bijlage \ref{Bijlagek6} is de implementatie hiervan te vinden.

Bij load scalability horen factor 6 (processes), 8 (concurrency) en 9 (disposability) van de 12-factor app methodologie.

\subsubsection{Functional scalability}
Definitie: In welke mate bestaande code moet worden aangepast zodra een nieuwe functionaliteit wordt toegevoegd aan het systeem.

\subsubsection{Onderhoudbaarheid}
Definitie: The degree of effectiveness and efficiency with which a product or system can be modified to improve it, correct it or adapt it to changes in environment, and in requirements


\subsection{conclusie}

\section{Technieken}
Deze paragraaf gaat over de deelvraag \enquote{\deeltechnieken}

\subsection{Schaalbaarheid}

\subsection{Onderhoudbaarheid}

\section{Verbeteringen}
Deze paragraaf gaat over de deelvraag \enquote{\deelverbetering}


\section{Implementatie}
Deze paragraaf gaat over de deelvraag \enquote{\deelimplementatie}


\section{Requirements}
Deze paragraaf gaat over de deelvraag \enquote{\deelrequirements}