\chapter{Reflectie}

\label{Reflectie}

\section{Literatuuronderzoek}
Het theoretisch kader heeft geholpen bij het definiëren van schaalbaarheid, onderhoudbaarheid en architectuur. De meest nuttige definities waren schaal- en onderhoudbaarheid, de definitie van architectuur is niet vaak voorgekomen in het onderzoek. Ik merkte dat schaalbaarheid erg lastig was om te definiëren, omdat er erg veel verschillende interpretaties van waren. Daarom heb ik een poging gedaan om een concretere definitie te creëren, het inbrengen van een nieuwe definitie (functionele, en niet-functionele schaalbaarheid) lijkt mij een succes. Er is aardig wat discussie over ontstaan, en ik ben van mening dat ik mijn steentje heb bijgedragen bij het creëren van een algemene definitie van schaalbaarheid.

Ik heb iets te veel literatuuronderzoek gedaan, waardoor de uiteindelijke onderzoeksresultaten daar onder geleden hebben. Als ik hier iets minder op had gefocust was het opgeleverde Proof-of-Value wellicht iets uitgebreider. Ook was de definitie van schaalbaarheid in de ogen van Developers.nl anders dan wat uit het literatuuronderzoek is gekomen. Wel heeft het theoretisch kader een daadwerkelijk stevige basis gelegd voor de rest van het onderzoek, het heeft voor een betere begripsvorming gezorgd waardoor ik helderder kon uitleggen waar het over ging. Verder was het combineren van de definities samen met de 15-Factor App erg nuttig.

\section{Uitvoering}
Ik had eerder moeten beginnen met het maken van concrete requirements. Dit had geholpen bij het onderzoek doordat het duidelijkere richtlijnen creëert. Onderzoek naar de technieken is goed verlopen, hierdoor kon de huidige situatie goed worden geëvalueerd en uiteindelijk kon er door middel van deze technieken worden gevalideerd of de situatie verbeterd is. Ik heb veel verbeteringen gevonden die mogelijk zijn om te implementeren. 

Niet al deze verbeteringen zijn geïmplementeerd tijdens het onderzoek, in verband met tijdstekort. Ik had iets te veel hooi op mijn vork genomen waardoor veel verbeteringen alleen bleven bij een aanbeveling. De implementatie heeft iets langer geduurd dan verwacht, maar ik ben trots op het feit dat het toch gelukt is. Het was geen makkelijke opdracht aangezien de methode van feature-environments niet frequent wordt gebruikt door anderen, en Policy-as-Code een vrij nieuwe techniek is met weinig documentatie. Dit heeft er voor gezorgd dat ik veel geleerd heb.

\section{Uitkomsten}
Over het algemeen ben ik van mening dat dit onderzoek zeker de hoofdvraag adequaat heeft beantwoord. Er zijn veel punten van verbetering gevonden en er kan goed mee verder worden gewerkt in de toekomst. De implementatie van feature-environments had ik niet voor ogen toen ik aan het onderzoek begon, dit heeft mij aangenaam verrast.
