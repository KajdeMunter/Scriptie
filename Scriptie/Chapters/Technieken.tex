\chapter{Technieken}

\label{Chapter3}

Dit hoofdstuk gaat over de deelvraag \enquote{\deeltechnieken}.

\section{ISO 25010}

ISO normen zijn wereldwijde standaarden, daarom is dit een ideale manier om kwaliteit te waarborgen. ISO-norm 25010 \parencite{ISO25010} is de opvolger van ISO-9126 en beschrijft software kwaliteitskenmerken in acht categorieën. Een categorie hiervan is onderhoudbaarheid. Elke categorie heeft een aantal subcategorieën, bij onderhoudbaarheid zijn dat \footnote{Vertaling van ISO-norm uit wikipedia: https://nl.wikipedia.org/wiki/ISO\_25010}:

\textbf{Modulariteit:} De mate waarin een systeem of computerprogramma opgebouwd is in losstaande componenten zodat wijzigingen van een component minimale impact hebben op andere componenten.

\textbf{Herbruikbaarheid:} De mate waarin een bestaand onderdeel gebruikt kan worden in meer dan één systeem of bij het bouwen van een nieuw onderdeel.

\textbf{Analyseerbaarheid:} De mate waarin het mogelijk is om effectief en efficiënt de impact, van een geplande verandering van één of meer onderdelen, op een product of systeem te beoordelen, om afwijkingen en/of foutoorzaken van een product vast te stellen of om onderdelen te identificeren die gewijzigd moeten worden.

\textbf{Wijzigbaarheid:} De mate waarin een product of systeem effectief en efficiënt gewijzigd kan worden zonder fouten of kwaliteitsvermindering tot gevolg.

\textbf{Testbaarheid:} De mate waarin effectief en efficiënt testcriteria vastgesteld kunnen worden voor een systeem, product of component en waarin tests uitgevoerd kunnen worden om vast te stellen of aan die criteria is voldaan.
\\\\
Om de applicatie \enquote{onderhoudbaar} te noemen moeten alle subcategorieën voldoende worden vervuld. Er moet een analyse worden uitgevoerd per subcategorie over eventuele tekortkomingen.


\section{De Twelve-Factor App}

A. Wiggins \parencite{12Factor} heeft een methodologie opgezet om moderne, schaalbare en onderhoudbare web-applicaties te bouwen. De methodologie past goed bij de probleemstelling, het minimaliseren van de kosten en tijd die het kost om nieuwe ontwikkelaars aan het project te laten werken en de gemakkelijkheid van het schalen. Daarnaast wilde Developers.nl graag weten of de 12-Factor app relevant kan zijn voor de huidige infrastructuur. Ook zorgt het voor structural scalability, draagbaarheid tussen uitvoeringsomgevingen, de mogelijkheid om te deployen op moderne cloud platformen en een minimale divergentie tussen development en productie waardoor CD gemakkelijk wordt om te implementeren.

De methodologie heeft 12 factoren (best-practices) die voor deze eigenschappen zorgen. Deze methodologie wordt vaak aangeraden door professionals, en is ook een methodologie die Developers.nl graag terug ziet in haar applicaties. Kritiek op de 12-factor app gaat voornamelijk over het feit dat het gelimiteerd is tot Heroku \parencite{AdaptingTwelveFactor}. Vijf jaar nadat Wiggins de 12-Factor App heeft opgesteld is K. Hoffman aan de slag gegaan met een boek genaamd \enquote{Beyond the 12-Factor App} \parencite{Beyond12Factor}. Dit boek heeft als doel om de 12 factoren concreter te definiëren en voegt daarnaast nog 3 extra factoren toe om applicaties in de cloud niet alleen te laten functioneren maar ook te laten gedijen. Hierdoor is de methodologie ook niet meer persé gericht op Heroku. Deze factoren zijn telemetry, security, en het concept \enquote{API first}. Eigenlijk is een betere benaming voor dit onderzoek dus de \enquote{Fifteen-Factor App}. In bijlage \ref{TabelFactors} is een gevolg samen met een uitleg bij elke factor geplaatst. Zodra een factor te maken heeft met onderhoudbaarheid is er een subcategorie van ISO 25010 onderhoudbaarheid bij geplaatst. Alle 15 factoren zullen worden meegenomen bij het behandelen van de deelvragen.

\section{Schaalbaarheids-controle}

In het onderzoek van Weinstock en Goodenough \parencite{OnSystemScalability} noemen zij dat het niet echt mogelijk is om te testen of een systeem schaalbaar is. Wel zijn er methoden om de schaalbaarheid te waarborgen:
\begin{itemize}
	\item Onderzoek de \enquote{performance curves} en karakteriseer deze met een Big O notatie. Hoe veranderen deze curves bij het aanpassen van een schaalstrategie? 
	
	\item Identificeer mechanismen om knelpunten aan het licht te brengen of waar aannames van het schaalbaarheids- ontwerp beginnen te worden geschonden. Deze knelpunten hebben vooral te maken met de eerste betekenis van Weinstock en Goodenough. Er moet gecontroleerd worden op de toenemende administratieve werkdruk, de \enquote{hard-coded} limieten op capaciteit, de user-interface en de complexiteitsgraad van algoritmen. De schaalbaarheids-aannames gaan over het onderzoeken hoe de uitbreiding van een systeem nieuwe problemen kan onthullen. Zodra een systeem zich uitbreidt is er een grotere kans op errors in de systeemconfiguratie, \enquote{zeldzame} errors komen vaker voor, is het belangrijk dat een probleem in het systeem gelokaliseerd blijft, en kan het een stuk complexer en lastiger worden om het systeem te begrijpen.
	
	\item Voer een SWOT analyse uit op de schaalbaarheids-strategie.
	\begin{itemize}
		\item \textbf{S}trengths (de soorten groei waar de strategie voor ontworpen is)
		\item \textbf{W}eaknesses (de soorten groei waar de strategie niet voor ontworpen is)
		\item \textbf{O}pportunities (mogelijke veranderingen in werklast of technologie die de strategie goed zou kunnen benutten)
		\item \textbf{T}hreats (mogelijke veranderingen in de werklast of technologie die de strategie in twijfel zouden kunnen trekken)
	\end{itemize}
\end{itemize}

Door het karakteriseren van de performance curves met een Big O notatie wordt voornamelijk de load scalability gewaarborgd. Het identificeren van de knelpunten en aannames samen met het uitvoeren van een SWOT analyse zorgt voor het waarborgen van de functionele schaalbaarheid.
 
\section{Conclusie}
De 12-Factor App is een methodologie die twaalf best-practices samenvoegt om moderne, schaal- en onderhoudbare web-applicaties te bouwen. Het boek \enquote{Beyond the 12-factor app} \parencite{Beyond12Factor} is hierop verder gegaan door nog een drietal factoren toe te voegen. Door een applicatie te evalueren op deze vijftien factoren, samen met de definitie van ISO-25010 \parencite{ISO25010} is te beoordelen of deze schaal- en onderhoudbaar is. Om de schaalbaarheid van een systeem te waarborgen zijn  van Weinstock en Goodenough \parencite{OnSystemScalability} een geschikte manier.