\chapter{Technieken}

\label{Chapter3}

Deze paragraaf gaat over de deelvraag \enquote{\deeltechnieken}.

\section{de "Twelve-Factor App"}

A. Wiggins \parencite{12Factor} heeft een methodologie opgezet om moderne, schaalbare en onderhoudbare web-applicaties te bouwen. De methodologie past goed bij de probleemstelling, het minimaliseren van de kosten en tijd die het kost om nieuwe ontwikkelaars aan het project te laten werken en de gemakkelijkheid van het schalen. Ook zorgt het voor structural scalability, draagbaarheid tussen uitvoeringsomgevingen, de mogelijkheid om te deployen op moderne cloud platformen en een minimale divergentie tussen development en productie waardoor CD gemakkelijk wordt om te implementeren.

De methodologie heeft 12 factoren die voor deze eigenschappen zorgen. Vijf jaar nadat Wiggins de 12 factor app heeft opgesteld is K. Hoffman aan de slag gegaan met een boek genaamd \enquote{Beyond the 12-factor app} \parencite{Beyond12Factor}. Dit boek heeft als doel om de 12 factoren concreter te definiëren en voegt daarnaast nog 3 extra factoren toe om applicaties in de cloud niet alleen te laten functioneren maar ook te laten gedijen. Deze factoren zijn telemetry, security, en het concept \enquote{API first}. In bijlage \ref{TabelFactors} is een gevolg samen met een uitleg bij elke factor geplaatst. Deze factoren zullen worden meegenomen bij het behandelen van de deelvragen.

\section{ISO 25010}

Kaas