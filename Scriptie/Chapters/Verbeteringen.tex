\chapter{Verbeteringen}

\label{Chapter5}

Dit hoofdstuk gaat over de deelvraag \enquote{\deelverbetering}.

\section{Feature-environments}

\textcolor{red}{TODO: reverse proxy, diagram, ansible, etc. etc.} % TODO feature environemnts

\section{Éen generieke infrastructuur}
One codebase, One application is Factor 1 van de 15-Factor App en zorgt voor herbruikbaarheid van ISO 25010. De huidige infrastructuur van de Website en het EMS wordt opgebouwd met Ansible. \enquote{Ansible is a radically simple IT automation engine that automates cloud provisioning, configuration management, application deployment, intra-service orchestration, and many other IT needs} \parencite{Ansible}. Kort gezegd is Ansible een Configuration Management (CM) tool.

Zoals de structuur nu is opgebouwd word er voor elke applicatie een apart stuk IaC geschreven. Om \textbf{modulariteit} en \textbf{herbruikbaarheid} van ISO 25010 \parencite{ISO25010} te verbeteren is het mogelijk om één centrale, algemene infrastructuur repository te maken waar installaties (stukken code dus) als Docker of databases en user-management kunnen worden hergebruikt voor meerdere applicaties. 

Andere IaC tools als Chef of Puppet, kunnen ook voor dit doeleinde worden gebruikt, maar aangezien Ansible al gebruikt wordt is het niet efficiënt om dit om te herschrijven naar iets anders. Het is mogelijk om nog een hoger level van abstractie toe te voegen door een IaC tool als Terraform te gebruiken om een machine in te richten.

\section{Policy-as-Code} % TODO: PAS OP Hier zit een requirement die niet in een variable zit
Developers.nl heeft in haar website al een aantal manieren om kwaliteit van code te waarborgen. Automatische unit, integration, functional en end-to-end tests. Er mist een manier om de infrastructuur te waarborgen op kwaliteit. Dit is een verplichtte requirement: \enquote{De oplossing moet een methode bevatten om kwaliteit van nieuwe toevoegingen aan de infrastructuur automatisch te waarborgen}. Door policies zijn er duidelijke standaarden waaraan moet worden voldaan. Policies kunnen in meerdere vormen voorkomen \parencite{WhyPaC}, waaronder:
\begin{itemize}
	\item \textbf{Compliance Policies.} Deze policies verzekeren dat nieuwe toevoegingen voldoen aan standaarden als bijvoorbeeld AVG of SOC.
	\item \textbf{Security Policies.} Deze policies verdedigen de integriteit van de infrastructuur door bijvoorbeeld te verzekeren dat bepaalde poorten niet open staan.
	\item \textbf{Operational Excellence.} Deze policies voorkomen uitval of verslechtering van geleverde services, bijvoorbeeld door nieuwe configuratie te valideren.
\end{itemize}

Vóór Policy-as-Code (PaC) werden deze policies opgeschreven door iemand en handmatig gecontroleerd. De -- nog vrij nieuwe -- techniek PaC richt zich op het automatiseren van dit proces door policies te kunnen definiëren in de vorm van code. 

De techniek zorgt ervoor dat configuratie getest kan worden op kwaliteit. Daarnaast komt het voordeel dat de policies opgeslagen kunnen worden in versiebeheer. Hierdoor kunnen de policies ook worden hergebruikt. Dit sluit goed aan met de huidige Infrastructure-as-Code (IaC) oplossing die Developers.nl gebruikt voor het inrichten van haar servers.

Het automatiseren van deze kwaliteitscontroles verhoogt de onderhoudbaarheid aanzienlijk. Als we uitgaan van de ISO-25010 definitie van onderhoudbaarheid \parencite{ISO25010} zorgt het omzetten van een handmatige naar een geautomatiseerde controle voor betere \textbf{analyseerbaarheid} op veranderingen van de systemen, omdat afwijkingen en/of fouten gemakkelijker worden vastgesteld. Daarnaast zijn policies in principe testcriteria, waardoor als gevolg ook de \textbf{testbaarheid} van de systemen stijgt bij het implementeren van policies as code. Ook zorgen policies voor \textbf{wijzigbaarheid}, aangezien het systeem gewijzigd kan worden zonder fouten of kwaliteitsverminderingen als gevolg omdat het simpelweg niet geïmplementeerd mag worden zodra een wijziging niet aan een policy voldoet. Bovendien zijn security policies handig voor het beschermen van de gevoelige data die het EMS bevat.

Er zijn twee technieken om PaC te implementeren. HashiCorps \enquote{Sentinel} en \enquote{Open Policy Agent} (OPA). In verband met de reden dat Sentinel closed-source is, is OPA de betere keuze. Dit past beter bij de bedrijfscultuur, slogan en budgetwensen van Developers.nl. Ook is Sentinel alleen toepasbaar op hashiCorp producten, waardoor de techniek een stuk beperkter is.

\section{Container Orchestration}
Om gebruik te maken van Factor 13 \textbf{(Concurrency)} moet de applicatie horizontaal en verticaal kunnen schalen. Dit heeft ook te maken met de should-requirement \enquote{De oplossing moet de website horizontaal kunnen laten schalen bij een toe- of afnemende hoeveelheid}. Omdat de systemen bij Developers.nl op Docker containers draaien moet er een manier zijn om deze te beheren. \enquote{Container orchestration} platforms helpen bij het automatiseren van alle aspecten behorend bij het beheren van containers. Dit doen zij door meerdere containers als één entiteit te beschouwen -- voor doeleinden van beschikbaarheid, schaalbaarheid, netwerken en de initiële containerimplementatie \parencite{ContainerOrchestration}.

Er zijn twee technieken voor container orchestration leidend in de context van Docker, namelijk Docker Swarm of Kubernetes (K8s). Over het algemeen is Swarm een stuk gemakkelijker en minder complex dan K8s. Dit zou betekenen dat als er rekening word gehouden met onderhoudbaarheid, swarm de beste keuze is om te gebruiken. Maar om de grootste hoeveelheid controle over de containers te hebben is K8s de juiste tool. Ook wordt K8s beter ondersteund door cloud providers doordat AWS, GCP, en Azure een speciale service bieden om K8s toe te passen. Dit heeft te maken met het feit dat de community van K8s ook een stuk groter is vergeleken met Swarm. Het is wel mogelijk om Swarm te gebruiken met de cloud services maar het er is geen out-of-the-box service zoals er bij K8s wel is. Als er gekeken wordt naar de wensen van Developers.nl en de scope van dit onderzoek is Swarm de meest passende keuze. De meeste prioriteit qua schalen gaat niet specifiek naar het automatiseren en managen van hoge hoeveelheden verkeer maar naar het draaien van meerderen omgevingen naast elkaar, om zo verschillende features apart te deployen of A/B te testen.

\section{Opschonen Docker volumes en images}
De images en volumes moeten automatisch worden opgeschoond.

\section{Logging \& Monitoring}
Logging en monitoring is een verplichtte requirement vanuit Developers.nl. Uit onderzoek in hoofdstuk \ref{Chapter4} blijkt dat Telemetry, Factor 14 van de 15-Factor app nog aan verbetering toe is. \parencite{Beyond12Factor} noemt drie verschillende categorieën van data om te monitoren in een applicatie:
\begin{itemize}
	\item Application performance monitoring (APM)
	\item Domain-specific telemetry
	\item Health and system logs
\end{itemize}

Om de juiste keuze van monitoring tool te maken is het belangrijk om te specificeren wat Developers.nl graag gemonitord ziet worden. Na een overleg met Jelle is besloten om te beginnen met APM als de hoeveelheid CPU/geheugen dat wordt gebruikt. Om dit te realiseren kan een tool als Prometheus samen met Grafana worden gebuikt, waardoor het erg simpel is om in de toekomst de hoeveelheid en soort data dat wordt gemonitord uit te breiden.

\section{Codecov}
Om de testbaarheid te verbeteren moet een tool worden gebruikt om de testcoverage van nieuwe toevoegingen te waarborgen. De meest prominente tool hiervoor is codecov. Het zorgt voor een duidelijk overzicht van de coverage tools en kan direct bij pull-requests nakijken of de nieuwe features wel voldoende zijn getest.

\section{Prioriteiten}
Om te beslissen welke verbeteringen worden geïmplementeerd zijn de verbeteringen gekoppeld aan de opgezette requirements.

\section{Conclusie}
Om de kenmerken modulariteit en herbruikbaarheid van ISO 25010 te verbeteren kan er een centrale infrastructuur IaC repository gemaakt worden met Ansible. Om de kenmerken analyseerbaarheid, testbaarheid en wijzigbaarheid te verbeteren kunnen policies worden afgedwongen door middel van PaC. Om de testbaarheid te waarborgen kan een tool als Codecov worden gebruikt.