\chapter{Verbeteringen}

\label{Chapter5}

Dit hoofdstuk gaat over de deelvraag \enquote{\deelverbetering}

\section{One codebase, One application}
One codebase, One application is Factor 1 van de 15-Factor App en zorgt voor herbruikbaarheid van ISO 25010. De huidige infrastructuur van de Website en het EMS wordt opgebouwd met Ansible. \enquote{Ansible is a radically simple IT automation engine that automates cloud provisioning, configuration management, application deployment, intra-service orchestration, and many other IT needs} \parencite{Ansible}. Dit doet Ansible door middel van een techniek genaamd \enquote{Infrastructure-as-Code}. Zoals de structuur nu is opgebouwd word er voor elke applicatie een apart stuk IaC geschreven. Om \textbf{modulariteit} en \textbf{herbruikbaarheid} van ISO 25010 \parencite{ISO25010} te verbeteren is het mogelijk om één centrale, algemene infrastructuur repository te maken waar installaties (stukken code dus) als Docker of databases en user-management kunnen worden hergebruikt voor meerdere applicaties. 

Andere IaC tools als Chef, Puppet, of Terraform kunnen ook voor dit doeleinde worden gebruikt, maar aangezien Ansible al gebruikt wordt is het niet efficiënt om dit om te herschrijven naar iets anders. \textcolor{red}{TODO: Terraform samen met Ansible?} % TODO: Terraform samen met ansible?

\section{Policy-as-Code}
Er zijn twee technieken om PaC te implementeren. HashiCorps \enquote{Sentinel} en \enquote{Open Policy Agent} (OPA). In verband met de reden dat Sentinel closed-source is, is er gekozen om OPA te gebruiken, dit past beter bij de bedrijfscultuur, slogan en budgetwensen van Developers.nl. Ook is Sentinel alleen toepasbaar op hashiCorp producten, waardoor de techniek een stuk beperkter is.

\section{Container Orchestration}
Er zijn twee technieken voor container orchestration leidend in de context van Docker, namelijk Docker Swarm of Kubernetes (K8s). Over het algemeen is Swarm een stuk gemakkelijker en minder complex dan K8s. Dit zou betekenen dat als er rekening word gehouden met onderhoudbaarheid, swarm de beste keuze is om te gebruiken. Maar om de grootste hoeveelheid controle over de containers te hebben is K8s de juiste tool. Ook wordt K8s beter ondersteund door cloud providers doordat AWS, GCP, en Azure een speciale service bieden om K8s toe te passen. Dit heeft te maken met het feit dat de community van K8s ook een stuk groter is vergeleken met Swarm. Het is wel mogelijk om Swarm te gebruiken met de cloud services maar het er is geen out-of-the-box service zoals er bij K8s wel is.

\section{Serverless computing}
In verband met de lage hoeveelheid verkeer op het EMS is \enquote{serverless computing} een goede oplossing om kosten te besparen.

\section{Cloud service providers}
Om te overwegen of een cloud provider bij de wensen van Developers.nl past worden de voor-en nadelen op een rijtje gezet:

\textbf{Voordelen cloud:} De kosten van cloud hosting zijn flexibel, er wordt alleen betaalt voor wat er daadwerkelijk gebruikt wordt, zolang er maar verstandig gebruik van wordt gemaakt. Dit betekend dat het mogelijk erg kostenefficiënt kan zijn voor Developers.nl aangezien er tijdens de maandelijkse \enquote{Tech Nights} piekmomenten zijn op de website, en er heel weinig verkeer is op het EMS. Een ander voordeel is dat er bijna een ongelimiteerde hoeveelheid opslagruimte beschikbaar is. Aangezien het CMS van de website dubbel functioneert als \enquote{file-server} en er veel foto's worden geüpload is het fijn dat er geen rekening hoeft worden gehouden met de hoeveelheid opslag. Bovendien worden software updates automatisch uitgevoerd, software als K8s kunnen al inbegrepen zijn bij de infrastructuur en het maakt schalen gemakkelijker. Aangezien cloud providers meer middelen voor beveiliging hebben wordt de veiligheid ook een stuk verbeterd.

\textbf{Voordelen traditioneel:} Ook al is cloud hosting meer kostenefficiënt is het toch mogelijk dat een web host goedkoper uitkomt. Zolang er maar geen hoge piekmomenten zijn in het verkeer. Dit is dus niet van toepassing op Developers.nl aangezien de \enquote{Tech Nights} of andere evenementen voor piekmomenten zorgen. \textcolor{red}{TODO: Huidige kosten? En deze tekst nagaan} % TODO: huidige kosten en deze tekst nagaan

\textbf{Nadelen cloud:} Cloud providers hebben de mogelijkheid voor technische problemen buiten de controle van klanten, waardoor het mogelijk is dat er downtime ontstaat. Ook kan het duurder uitpakken zodra er niet goed word omgegaan met de schaalstrategie of benodigde rekenkracht.

\textbf{Nadelen traditioneel:} De mogelijkheid bestaat dat er meer kosten worden gemaakt dan nodig is. Ook is shared-hosting een risico omdat zodra een andere klant veel rekenkracht opeist de kans ontstaat dat de prestaties dalen.\\

Er zijn veel verschillende cloud providers, waarvan de grootste Amazon Web Services (AWS), Microsoft Azure en Google Cloud Platform (GCP) zijn. \textcolor{red}{TODO: kiezen welke.. kosten nagaan? Veel werk?} % TODO: Kiezen welke

\section{Docker cleanup}
De images en volumes moeten automatisch worden opgeschoond.

\section{Telemetry}
Factor 14.

\section{Conclusie}
Om de kenmerken modulariteit en herbruikbaarheid van ISO 25010 te verbeteren kan er een centrale infrastructuur IaC repository gemaakt worden met Ansible. Om de kenmerken analyseerbaarheid, testbaarheid en wijzigbaarheid te verbeteren kunnen policies worden afgedwongen door middel van PaC.