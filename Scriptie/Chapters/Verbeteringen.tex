\chapter{Verbeteringen}

\label{Chapter5}

Dit hoofdstuk gaat over de deelvraag \enquote{\deelverbetering}. Gebaseerd op de conclusie uit hoofdstuk \ref{Chapter4} en de requirements van Developers.nl zijn verschillende technieken onderzocht om de gevonden verbeterpunten te verbeteren. Daarna zijn deze gevonden technieken op prioriteit geordend door middel van de requirements. 

\section{Feature-environments}
Met feature-environments is het mogelijk om een staging-omgeving te creëren voor elke individuele feature branch. Dit betekent dat de product owners en developers elke feature afzonderlijk van andere features kunnen testen, om deze daarna naar productie te deployen. De feature-environment workflow is niet erg conventioneel of populair, aannemelijk omdat het niet gemakkelijk is om te implementeren. De meest relevante informatie voor dit onderzoek is een blog van Christian Lüdemann \parencite{FeatureEnvs}, hoewel de implementatie niet past bij de huidige situatie van Developers.nl. Uit onderzoek over verschillende technieken zijn de volgende opties gekomen:
\begin{itemize}
	\item Een Kubernetes cluster met individuele namespaces voor elke environment.
	\item Tools als \texttt{https://platform.sh/} of \texttt{https://continuouspipe.io/}.
	\item Een dynamische reverse proxy die requests naar verschillende docker netwerken stuurt.
	\item Een dynamische reverse proxy die requests naar verschillende aparte VM's stuurt die gemanaged worden met Proxmox, of een cloud provider.
\end{itemize}

Een kubernetes cluster net namespaces valt af, dit heeft te maken met het onderhoudswerk van een cluster. Meer hierover is te lezen in paragraaf \ref{ContainerOrchestration}. Ook Docker Swarm is geen optie, Swarm heeft geen equivalent van kubernetes namespaces en er kunnen niet meerdere nodes op één machine staan. Tools als Platform.sh of Continuouspipe.io hebben een geïntegreerde oplossing voor feature environments, maar voegen veel overhead toe. Bovendien kost Platform.sh geld, waar Developers.nl het niet voor over heeft.

Er zijn twee opties voor het maken van feature environments met een reverse proxy. Het gebruik maken van aparte VM's versus het aanmaken van docker netwerken. Het zusterbedrijf van Developers.nl -- gemvision -- maakt gebruik van Proxmox om hun Virtual Machines te beheren. Hoewel Proxmox een goede tool is om de VM's te beheren, is het echt gebaseerd op enterprise settings. Dit maakt het lastiger voor andere ontwikkelaars om goed gebruik te maken van deze tool. Aangezien Docker netwerken al geïntegreerd zijn in Docker voegt deze oplossing veel minder overhead toe. Dit is dus de beste optie. Er zijn verschillende reverse proxies. Relevante voor dit project zijn:
\begin{itemize}
	\item Traefik
	\item jwilder/Nginx-Proxy
	\item HAProxy	
\end{itemize}

Elke reverse proxy heeft zijn voor- en nadelen. De Website en het EMS maken al gebruik van Nginx als FastCGI webserver voor PHP-FPM. Daarom is nginx-proxy een interessante optie\footnote{https://github.com/jwilder/nginx-proxy}. Nginx-proxy is een abstractielaag boven Nginx die gebruik maakt van de Docker socket om op die manier requests naar de juiste plek te sturen. Het nadeel is dat Nginx hier niet expliciet voor bedoeld is, waardoor een aantal tekortkomingen aanwezig zijn, zoals ondersteuning voor Docker Swarm\footnote{https://github.com/jwilder/nginx-proxy/issues/927}. Traefik werkt goed samen met Docker (swarm), aangezien Traefik ingebouwde service-discovery heeft voor docker containers en Let's Encrypt. Bovendien heeft Traefik een minder steile learning curve door de aanwezige documentatie. Het nadeel is dat het meer overhead creëert doordat het een compleet aparte en nieuwe tool is die moet worden toegevoegd. Dit nadeel is ook aanwezig bij HAProxy. HAProxy is snel en capabel voor load balancing, maar is complex om op te zetten zodra het op feature environments toekomt, omdat er geen ingebouwde service-discovery aanwezig is. 

In het geval van Developers.nl is Traefik de beste optie, omdat het de minste complexiteit bevat en goed samen met Docker en Docker Swarm werkt. Een nadeel van de service-discovery oplossingen is dat de Docker Socket moet worden geëxposeerd. Dit is een groot beveiligingslek\footnote{https://github.com/containous/traefik/issues/4174}. Zodra een hacker met kwaadaardige intenties Traefik weet te kapen betekent dat dat de hacker root toegang heeft op de host machine. Een oplossing hiervoor is het exposen van de Docker Socket door het Transmission Control Protocol (TCP) en deze te beveiligen door middel van Transport Layer Security (TLS).

In bijlage \ref{FeedbackFeatureEnvironments} zijn Slack conversaties met Jelle te vinden die te maken hebben met de feature environments, en over het beveiligen van de Docker socket.

\section{Éen generieke infrastructuur}
One codebase, One application is Factor 1 van de 15-Factor App en zorgt voor herbruikbaarheid van ISO 25010. De huidige infrastructuur van de Website en het EMS wordt opgebouwd met Ansible. \enquote{Ansible is a radically simple IT automation engine that automates cloud provisioning, configuration management, application deployment, intra-service orchestration, and many other IT needs} \parencite{Ansible}. Kort gezegd is Ansible een Configuration Management (CM) tool.

Zoals de structuur nu is opgebouwd word er voor elke applicatie een apart stuk IaC geschreven. Om modulariteit en herbruikbaarheid van ISO 25010 \parencite{ISO25010} te verbeteren is het mogelijk om één centrale, algemene infrastructuur repository te maken waar installaties (stukken code dus) als Docker of databases en user-management kunnen worden hergebruikt voor meerdere applicaties. 

Andere IaC tools als Chef of Puppet, kunnen ook voor dit doeleinde worden gebruikt, maar aangezien Ansible al gebruikt wordt is het niet efficiënt om dit om te herschrijven naar iets anders. Het is mogelijk om nog een hoger level van abstractie toe te voegen door een IaC tool als Terraform te gebruiken om een machine in te richten.

\section{Policy-as-Code} % TODO: PAS OP Hier zit een requirement die niet in een variable zit
Developers.nl heeft in haar website al een aantal manieren om kwaliteit van code te waarborgen. Automatische unit, integration, functional en end-to-end tests. Er mist een manier om de infrastructuur te waarborgen op kwaliteit. Dit is een verplichtte requirement: \enquote{De oplossing moet een methode bevatten om kwaliteit van nieuwe toevoegingen aan de infrastructuur automatisch te waarborgen}. Door policies zijn er duidelijke standaarden waaraan moet worden voldaan. Policies kunnen in meerdere vormen voorkomen \parencite{WhyPaC}, waaronder:
\begin{itemize}
	\item \textbf{Compliance Policies.} Deze policies verzekeren dat nieuwe toevoegingen voldoen aan standaarden als bijvoorbeeld AVG of SOC.
	\item \textbf{Security Policies.} Deze policies verdedigen de integriteit van de infrastructuur door bijvoorbeeld te verzekeren dat bepaalde poorten niet open staan.
	\item \textbf{Operational Excellence.} Deze policies voorkomen uitval of verslechtering van geleverde services, bijvoorbeeld door nieuwe configuratie te valideren.
\end{itemize}

Vóór Policy-as-Code (PaC) werden deze policies opgeschreven door iemand en handmatig gecontroleerd. De -- nog vrij nieuwe -- techniek PaC richt zich op het automatiseren van dit proces door policies te kunnen definiëren in de vorm van code. 

De techniek zorgt ervoor dat configuratie getest kan worden op kwaliteit. Daarnaast komt het voordeel dat de policies opgeslagen kunnen worden in versiebeheer. Hierdoor kunnen de policies ook worden hergebruikt. Dit sluit goed aan met de huidige Infrastructure-as-Code (IaC) oplossing die Developers.nl gebruikt voor het inrichten van haar servers.

Het automatiseren van deze kwaliteitscontroles verhoogt de onderhoudbaarheid aanzienlijk. Als we uitgaan van de ISO-25010 definitie van onderhoudbaarheid \parencite{ISO25010} zorgt het omzetten van een handmatige naar een geautomatiseerde controle voor betere \textbf{analyseerbaarheid} op veranderingen van de systemen, omdat afwijkingen en/of fouten gemakkelijker worden vastgesteld. Daarnaast zijn policies in principe testcriteria, waardoor als gevolg ook de \textbf{testbaarheid} van de systemen stijgt bij het implementeren van policies as code. Ook zorgen policies voor \textbf{wijzigbaarheid}, aangezien het systeem gewijzigd kan worden zonder fouten of kwaliteitsverminderingen als gevolg omdat het simpelweg niet geïmplementeerd mag worden zodra een wijziging niet aan een policy voldoet. Bovendien zijn security policies handig voor het beschermen van de gevoelige data die het EMS bevat.

Er zijn twee technieken om PaC te implementeren. HashiCorps \enquote{Sentinel} en \enquote{Open Policy Agent} (OPA). In verband met de reden dat Sentinel closed-source is, is OPA de betere keuze. Dit past beter bij de bedrijfscultuur, slogan en budgetwensen van Developers.nl. Ook is Sentinel alleen toepasbaar op hashiCorp producten, waardoor de techniek een stuk beperkter is.

\section{Container Orchestration}
\label{ContainerOrchestration}
Om gebruik te maken van Factor 13 \textbf{(Concurrency)} moet de applicatie horizontaal en verticaal kunnen schalen. Dit heeft ook te maken met de should-requirement \enquote{De oplossing moet de website horizontaal kunnen laten schalen bij een toe- of afnemende hoeveelheid}. Omdat de systemen bij Developers.nl op Docker containers draaien moet er een manier zijn om deze te beheren. \enquote{Container orchestration} platforms helpen bij het automatiseren van alle aspecten behorend bij het beheren van containers. Dit doen zij door meerdere containers als één entiteit te beschouwen -- voor doeleinden van beschikbaarheid, schaalbaarheid, netwerken en de initiële containerimplementatie \parencite{ContainerOrchestration}.

Er zijn twee technieken voor container orchestration leidend in de context van Docker, namelijk Docker Swarm of Kubernetes (K8s). Over het algemeen is Swarm een stuk gemakkelijker en minder complex dan K8s. Dit zou betekenen dat als er rekening word gehouden met onderhoudbaarheid, swarm de beste keuze is om te gebruiken. Maar om de grootste hoeveelheid controle over de containers te hebben is K8s de juiste tool. Ook wordt K8s beter ondersteund door cloud providers doordat AWS, GCP, en Azure een speciale service bieden om K8s toe te passen. Dit heeft te maken met het feit dat de community van K8s ook een stuk groter is vergeleken met Swarm. Het is wel mogelijk om Swarm te gebruiken met de cloud services maar het er is geen out-of-the-box service zoals er bij K8s wel is. 

Als er gekeken wordt naar de wensen van Developers.nl en de scope van dit onderzoek is Swarm de meest passende keuze. De meeste prioriteit qua schalen gaat niet specifiek naar het automatiseren en managen van hoge hoeveelheden verkeer maar naar het draaien van meerderen omgevingen naast elkaar, om zo verschillende features apart te deployen of A/B te testen.

\section{Opschonen Docker images}
De Docker images moeten automatisch worden opgeschoond. Dit is mogelijk om periodiek uit te voeren met een \texttt{systemd} timer of een \texttt{cron job}, maar omdat Developers.nl al gebruik maakt van Ansible in hun websites kan het gemakkelijk geïmplementeerd worden met een Ansible taak die bij elke deployment gebruik maakt van de \texttt{docker\_prune} module.

\section{Logging \& Monitoring}
Logging en monitoring is een verplichtte requirement vanuit Developers.nl. Uit onderzoek in hoofdstuk \ref{Chapter4} blijkt dat Telemetry, Factor 14 van de 15-Factor app nog aan verbetering toe is. \parencite{Beyond12Factor} noemt drie verschillende categorieën van data om te monitoren in een applicatie:
\begin{itemize}
	\item Application performance monitoring (APM)
	\item Domain-specific telemetry
	\item Health and system logs
\end{itemize}

Om de juiste keuze van monitoring tool te maken is het belangrijk om te specificeren wat Developers.nl graag gemonitord ziet worden. Na een overleg met Jelle is besloten om te beginnen met APM als de hoeveelheid CPU/geheugen dat wordt gebruikt. Om dit te realiseren kan een tool als Prometheus samen met Grafana worden gebuikt, waardoor het erg simpel is om in de toekomst de hoeveelheid en soort data dat wordt gemonitord uit te breiden.

\section{Codecov}
Om de testbaarheid te verbeteren moet een tool worden gebruikt om de testcoverage van nieuwe toevoegingen te waarborgen. De meest prominente tool hiervoor is codecov. Het zorgt voor een duidelijk overzicht van de coverage tools en kan direct bij pull-requests nakijken of de nieuwe features wel voldoende zijn getest.

\section{Prioriteiten}
Om te beslissen welke verbeteringen geïmplementeerd worden voor dit onderzoek worden ze op prioriteit geordend door middel van ze onder de requirements te verdelen.

% Feature Environments
% Policy-as-code
% Opschonen docker images
% Logging \& Monitoring
% Codecov
% Container orchestration
% Generieke infrastructuur

\subsubsection{Must-requirements:}
\begin{itemize}
	\item De oplossing moet méér dan twee unieke instanties van de website naast elkaar kunnen draaien.
	\begin{itemize}
		\item Feature Environments
	\end{itemize}
	
	\item De oplossing moet moet unieke instanties van de website automatisch kunnen aanmaken.
	\begin{itemize}
		\item Feature Environments
	\end{itemize}

	\item De oplossing moet een methode bevatten om kwaliteit van nieuwe toevoegingen aan de infrastructuur automatisch te waarborgen.
	\begin{itemize}
		\item Policy-as-code
		\item Codecov
	\end{itemize}

	\item Minimaal één monitoring tool voor het monitoren van performance.
	\begin{itemize}
		\item Logging \& Monitoring
	\end{itemize}

	\item De oplossing moet voldoen aan één of meerdere kwaliteitsstandaarden.
	\begin{itemize}
		\item Opschonen Docker images
		\item Logging \& monitoring
	\end{itemize}
\end{itemize}

\subsubsection{Should-requirements:}
\begin{itemize}
	\item De oplossing moet generiek genoeg zijn zodat meerdere applicaties er gebruik van kunnen maken.
	\begin{itemize}
		\item Generieke infrastructuur
	\end{itemize}

	\item De oplossing moet de website horizontaal kunnen laten schalen bij een toe- of afnemende hoeveelheid verkeer. De efficiëntie moet 1:1 zijn. Bijvoorbeeld: Éen extra instantie moet 50\% van het verkeer opvangen.
	\begin{itemize}
		\item Container orchestration
	\end{itemize}
\end{itemize}

\section{Conclusie}
Om de kenmerken modulariteit en herbruikbaarheid van ISO 25010 te verbeteren kan er een centrale infrastructuur IaC repository gemaakt worden met Ansible. Om de kenmerken analyseerbaarheid, testbaarheid en wijzigbaarheid te verbeteren kunnen policies worden afgedwongen door middel van PaC. Om de testbaarheid te waarborgen kan een tool als Codecov worden gebruikt.