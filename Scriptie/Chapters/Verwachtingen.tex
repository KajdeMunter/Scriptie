\chapter{Verwachtingen}

\label{Verwachtingen}

Dit hoofdstuk gaat over de deelvraag \enquote{\deelverwachtingen}. Om de verwachtingen duidelijk in kaart te brengen zijn discussies gevoerd met Jelle van de Haterd; stagebegeleider en kennisambassadeur DevOps. Notulen van deze discussie zijn te vinden in bijlage \ref{FeedbackRequirements}. De requirements zijn gebaseerd op de huidige situatie van de website, in hoofdstuk \ref{Chapter4} wordt hier nader op ingegaan.

\section{Hoe ziet Developers.nl onderhoudbaarheid?}

Om de workflow sneller te laten verlopen moeten zo veel mogelijk processen geautomatiseerd worden. Hieronder valt voornamelijk het automatiseren van de kwaliteitswaarborging. Het is belangrijk dat een nieuwe toevoeging of applicatie aan een aantal standaarden voldoet, zodat ontwikkelaars snel aan de slag kunnen. Developers.nl vind ten opzichte van daadwerkelijke code voornamelijk dat de infrastructuur aan verbetering toe is.

Developers.nl wil een nieuwe toevoeging of eventueel een volledig nieuwe applicatie kunnen bedenken en deze zo snel mogelijk geïmplementeerd zien. De workflow qua integratie en deployment moet verder geoptimaliseerd worden om Developers.nl deze snelheid te beloven.

\section{Hoe ziet Developers.nl schaalbaarheid?}

Developers.nl wil de voortgang van nieuwe toevoegingen of applicaties nauw kunnen volgen, zodat de kwaliteit hiervan op tijd gevalideerd kan worden. Dit heeft te maken met de agile werkmethodiek die Developers.nl gebruikt voor het ontwikkelen van haar interne applicaties.

Om de snelheid te verhogen waarop ontwikkelaars hun toevoegingen kunnen laten zien, ziet Developers.nl graag het concept van \enquote{Feature environments}. Dit wil zeggen dat er voor elke aangemaakte git branch gelijk gedeployed wordt naar een aparte omgeving, die dan te bekijken is door de Product Owner (PO). Dit betekent dat er dus meerdere (verschillende) instanties van de applicatie naast elkaar moeten kunnen draaien. Daarnaast ziet Developers.nl graag de mogelijkheid om rekenkracht van de server te verdelen over deze instanties, waardoor er dus horizontaal geschaald moet kunnen worden. Developers.nl prioriteert de feature environments boven het verdelen van de rekenkracht, omdat het implementeren van feature environments \enquote{twee vliegen in één klap} is. Dit zorgt namelijk voor onderhoudbaarheid én het betekent dat de oplossing schaalbaar is.

\section{Waar wilt Developers.nl meer over te weten komen?}

Developers.nl wil weten of de best-practices van de 12-Factor App toepasselijk zijn op de huidige infrastructuur. Ook wil Developers.nl meer te weten komen over verschillende standaarden die zij kunnen opvolgen om de kwaliteit van hun infrastructuur te waarborgen.

\section{Wat zijn de concrete requirements waar de oplossing aan moet voldoen?}

Om schaalbaarheid te realiseren moet de onderhoudbaarheid van de infrastructuur ook op een voldoende niveau zijn, hier moeten kwaliteitsstandaarden voor onderzocht worden. Daarnaast moet de oplossing zo generiek mogelijk zijn, en dus voor meerdere applicaties toe te passen zijn. Dit betekent dus ook dat de oplossing geschikt moet zijn voor applicaties met veel verkeer.

Om requirements op te stellen wordt gebruik gemaakt van de MoSCoW-methode. Deze afkorting staat voor: \textbf{M}ust haves, \textbf{S}hould haves, \textbf{C}ould haves, en \textbf{W}on't haves \parencite{Moscow}. Deze methode helpt bij het opstellen van prioriteiten om zo te beslissen wat in de scope van dit onderzoek valt.

\subsubsection{Must haves}
\begin{itemize}
	\item De oplossing moet méér dan twee unieke instanties van de website naast elkaar kunnen draaien.
	\item De oplossing moet moet unieke instanties van de website automatisch kunnen aanmaken.
	\item De oplossing moet een methode bevatten om kwaliteit van nieuwe toevoegingen aan de infrastructuur automatisch te waarborgen.
	\item Minimaal één monitoring tool voor het monitoren van performance.
	\item De oplossing moet voldoen aan één of meerdere kwaliteitsstandaarden.
\end{itemize}

\subsubsection{Should haves}
\begin{itemize}
	\item De oplossing moet generiek genoeg zijn zodat meerdere applicaties er gebruik van kunnen maken.
	\item De oplossing moet de website horizontaal kunnen laten schalen bij een toe- of afnemende hoeveelheid verkeer. De efficiëntie moet 1:1 zijn. Bijvoorbeeld: Éen extra instantie moet 50\% van het verkeer opvangen.
\end{itemize}

\subsubsection{Could haves}
\begin{itemize}
	\item De oplossing moet de website automatisch laten schalen.
	\item Er moet onderzoek gedaan worden naar cloud providers, en of dit een goede verbetering is.
	\item Er moet onderzoek gedaan worden naar serverless computing, en of dit een goede verbetering is.
\end{itemize}

\subsubsection{Won't haves}
\begin{itemize}
	\item Een \enquote{boilerplate} voor het opzetten van nieuwe (schaal- en onderhoudbare) projecten.
\end{itemize}

\section{Conclusie}
Developers.nl wilt onderhoudbaarheid bereiken door kwaliteitsstandaarden af te dwingen. Ook wilt Developers.nl twee soorten schaalbaarheid. Eén in de vorm van het deployen van verschillende branches in aparte omgevingen, en één in de vorm van horizontaal schalen om meer verkeer aan te kunnen. Deze twee soorten kunnen niet gerealiseerd worden zonder de onderhoudbaarheid te waarborgen. Er zijn vijf must-have requirements:

\begin{itemize}
	\item De oplossing moet méér dan twee unieke instanties van de website naast elkaar kunnen draaien.
	\item De oplossing moet moet unieke instanties van de website automatisch kunnen aanmaken.
	\item De oplossing moet een methode bevatten om kwaliteit van nieuwe toevoegingen aan de infrastructuur automatisch te waarborgen.
	\item Minimaal één monitoring tool voor het monitoren van performance.
	\item De oplossing moet voldoen aan één of meerdere kwaliteitsstandaarden.
\end{itemize}