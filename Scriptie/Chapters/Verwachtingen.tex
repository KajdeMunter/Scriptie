\chapter{Verwachtingen}

\label{Verwachtingen}

Dit hoofdstuk gaat over de deelvraag \enquote{\deelverwachtingen}. Om de verwachtingen duidelijk in kaart te brengen zijn discussies gevoerd met Jelle van de Haterd; stagebegeleider en kennisambassadeur DevOps. Notulen van deze discussie zijn te vinden in bijlage \ref{FeedbackRequirements}.

\section{Hoe ziet Developers.nl onderhoudbaarheid?}

Om de workflow sneller te laten verlopen moeten zo veel mogelijk processen geautomatiseerd worden. Hieronder valt voornamelijk het automatiseren van de kwaliteitswaarborging. Het is belangrijk dat een nieuwe toevoeging of applicatie aan een aantal standaarden voldoet, zodat ontwikkelaars snel aan de slag kunnen. Jelle vind ten opzichte van daadwerkelijke code voornamelijk dat de infrastructuur aan verbetering toe is.

Developers.nl als product owner wil een nieuwe toevoeging of zelfs volledig nieuwe applicatie kunnen bedenken en deze zo snel mogelijk geïmplementeerd zien. De workflow qua integratie en deployment moet verder geoptimaliseerd worden om Developers.nl deze snelheid te beloven.

\section{Hoe ziet Developers.nl schaalbaarheid?}

Developers.nl als product owner wil de voortgang van nieuwe toevoegingen of applicaties nauw kunnen volgen, zodat de kwaliteit hiervan op tijd gevalideerd kan worden. Dit heeft te maken met de agile werkmethodiek die Developers.nl gebruikt voor het ontwikkelen van haar interne applicaties.

Om de snelheid te verhogen waarop ontwikkelaars hun toevoegingen kunnen laten zien, ziet Developers.nl graag het concept van \enquote{Feature environments}. Dit wil zeggen dat er per aangemaakte git branch gelijk gedeployed wordt naar een aparte omgeving, die dan te bekijken is door de product owner. Dit betekent dat er dus meerdere (verschillende) instanties van de applicatie naast elkaar moeten kunnen draaien. Daarnaast ziet Developers.nl graag de mogelijkheid om rekenkracht van de server te verdelen over deze instanties, waardoor er dus horizontaal geschaald moet kunnen worden.

\section{Waar wilt Developers.nl meer over te weten komen?}

Developers.nl wil weten of de best-practices van de 12-factor app toepasselijk zijn op de huidige infrastructuur. Ook wil Developers.nl meer te weten komen over verschillende standaarden die zij kunnen opvolgen om de kwaliteit van hun infrastructuur te waarborgen.

\section{Wat zijn de concrete requirements waar de oplossing aan moet voldoen?}

Om schaalbaarheid te realiseren moet de onderhoudbaarheid van de infrastructuur ook op een voldoende niveau zijn, hier \textbf{moeten} kwaliteitsstandaarden voor onderzocht worden. Daarnaast moet de oplossing zo generiek mogelijk zijn, en dus voor meerdere applicaties toe te passen zijn. Dit betekend dus ook dat de oplossing geschikt moet zijn voor applicaties met veel verkeer. Concreet moet de oplossing aan het eind van de stageperiode;

\begin{itemize}
	\item Méér dan twee unieke instanties van een applicatie naast elkaar kunnen draaien, en deze automatisch kunnen aanmaken.
	\item Generiek genoeg zijn zodat meerdere applicaties er gebruik van kunnen maken.
	\item Een manier bevatten om kwaliteit van nieuwe toevoegingen aan de infrastructuur automatisch te waarborgen.
	\item Horizontaal kunnen schalen bij een toe- of afnemende hoeveelheid verkeer.
\end{itemize}

\section{Conclusie}

Developers.nl wilt onderhoudbaarheid bereiken door kwaliteitsstandaarden af te dwingen. Ook wilt Developers.nl twee soorten schaalbaarheid. Eén in de vorm van het deployen van verschillende branches in aparte omgevingen, en één in de vorm van horizontaal schalen om meer verkeer aan te kunnen. Deze twee soorten kunnen niet gerealiseerd worden zonder de onderhoudbaarheid te waarborgen. Er zijn vier concrete requirements:

\begin{itemize}
	\item Méér dan twee unieke instanties van een applicatie naast elkaar kunnen draaien, en deze automatisch kunnen aanmaken.
	\item Generiek genoeg zijn zodat meerdere applicaties er gebruik van kunnen maken.
	\item Een manier bevatten om kwaliteit van nieuwe toevoegingen aan de infrastructuur automatisch te waarborgen.
	\item Horizontaal kunnen schalen bij een toe- of afnemende hoeveelheid verkeer.
\end{itemize}
