\chapter{Verwachtingen}

\label{Verwachtingen}

Dit hoofdstuk gaat over de deelvraag \enquote{\deelverwachtingen}

\section{Hoe ziet Developers.nl onderhoudbaarheid?}

Om de workflow sneller te laten verlopen moeten zo veel mogelijk processen geautomatiseerd worden. Hieronder valt voornamelijk het automatiseren van de kwaliteitswaarborging. Het is belangrijk dat een nieuwe toevoeging of applicatie aan een aantal standaarden voldoet. 

\section{Hoe ziet Developers.nl schaalbaarheid?}

Daarnaast wil Developers.nl als product owner deze nieuwe toevoegingen of applicatie zo snel mogelijk handmatig kunnen valideren. Om dit te realiseren moet er voor individuele toevoegingen naast elkaar verschillende omgevingen kunnen worden opgezet die te bekijken zijn door de product owners.

\section{Waar wilt Developers.nl meer over te weten komen?}

Developers.nl wilt weten of de best-practices van de 12-factor app toepasselijk zijn op de huidige infrastructuur. Ook wilt Developers.nl meer te weten komen over verschillende standaarden die zij kunnen opvolgen.

\section{Wat zijn de concrete requirements waar de oplossing aan moet voldoen?}

Om schaalbaarheid te realiseren moet de onderhoudbaarheid van de infrastructuur ook op een voldoende niveau zijn, hier \textbf{moeten} kwaliteitsstandaarden voor onderzocht worden. Daarnaast moet de oplossing zo generiek mogelijk zijn, en dus voor meerdere applicaties toe te passen zijn. Dit betekend dus ook dat de oplossing geschikt moet zijn voor applicaties met veel verkeer. Concreet moet de oplossing aan het eind van de stageperiode;

\begin{itemize}
	\item Méér dan twee unieke instanties van een applicatie naast elkaar kunnen draaien, en deze automatisch kunnen aanmaken.
	\item Generiek genoeg zijn zodat meerdere applicaties er gebruik van kunnen maken.
	\item Een manier bevatten om kwaliteit van nieuwe toevoegingen aan de infrastructuur automatisch te waarborgen.
	\item Horizontaal kunnen schalen bij een toe- of afnemende hoeveelheid verkeer.
\end{itemize}

\section{Conclusie}

Boter, Kaas en Eieren.