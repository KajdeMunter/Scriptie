%%%%%%%%%%%%%%%%%%%%%%%%%%%%%%%%%%%%%%%%%
% Masters/Doctoral Thesis 
% LaTeX Template
% Version 2.5 (27/8/17)
%
% This template was downloaded from:
% http://www.LaTeXTemplates.com
%
% Version 2.x major modifications by:
% Vel (vel@latextemplates.com)
%
% This template is based on a template by:
% Steve Gunn (http://users.ecs.soton.ac.uk/srg/softwaretools/document/templates/)
% Sunil Patel (http://www.sunilpatel.co.uk/thesis-template/)
%
% Template license:
% CC BY-NC-SA 3.0 (http://creativecommons.org/licenses/by-nc-sa/3.0/)
%
%%%%%%%%%%%%%%%%%%%%%%%%%%%%%%%%%%%%%%%%%

%-------------------------------------------------
%	PACKAGES AND OTHER DOCUMENT CONFIGURATIONS
%-------------------------------------------------

\documentclass[
11pt, % The default document font size, options: 10pt, 11pt, 12pt
%oneside, % Two side (alternating margins) for binding by default, uncomment to switch to one side
english, % ngerman for German
singlespacing, % Single line spacing, alternatives: onehalfspacing or doublespacing
%draft, % Uncomment to enable draft mode (no pictures, no links, overfull hboxes indicated)
%nolistspacing, % If the document is onehalfspacing or doublespacing, uncomment this to set spacing in lists to single
liststotoc, % Uncomment to add the list of figures/tables/etc to the table of contents
%toctotoc, % Uncomment to add the main table of contents to the table of contents
%parskip, % Uncomment to add space between paragraphs
%nohyperref, % Uncomment to not load the hyperref package
headsepline, % Uncomment to get a line under the header
%chapterinoneline, % Uncomment to place the chapter title next to the number on one line
%consistentlayout, % Uncomment to change the layout of the declaration, abstract and acknowledgements pages to match the default layout
]{formatting} % The class file specifying the document structure

\usepackage[utf8]{inputenc} % Required for inputting international characters
\usepackage[T1]{fontenc} % Output font encoding for international characters
\usepackage{mathpazo} % Use the Palatino font by default
\usepackage[ddmmyyyy]{datetime}
\usepackage[backend=biber,style=ieee]{biblatex} % Use bibtext and IEEE reference style
\usepackage[autostyle=true]{csquotes} % Required to generate language-dependent quotes in the bibliography
\usepackage{minted} % Code highlighting


\addbibresource{literatuurlijst.bib} % The filename of the bibliography

%----------------------------
%	MARGIN SETTINGS
%----------------------------

\geometry{
	paper=a4paper, % Change to letterpaper for US letter
	inner=2.5cm, % Inner margin
	outer=3.8cm, % Outer margin
	bindingoffset=.5cm, % Binding offset
	top=1.5cm, % Top margin
	bottom=1.5cm, % Bottom margin
	%showframe, % Uncomment to show how the type block is set on the page
}

%----------------------------
%	THESIS INFORMATION
%----------------------------

\thesistitle{Schaal- en onderhoudbaarheid Developers.nl} % Your thesis title, this is used in the title and abstract, print it elsewhere with \ttitle
\technicalSupervisor{Tanja Ubert} % Your supervisor's name, this is used in the title page, print it elsewhere with \techsupname
\skillsSupervisor{Judith Lemmens} % skillssupname

\examiner{} % Your examiner's name, this is not currently used anywhere in the template, print it elsewhere with \examname
\degree{Informatica} % Your degree name, this is used in the title page and abstract, print it elsewhere with \degreename
\author{Kaj de Munter} % Your name, this is used in the title page and abstract, print it elsewhere with \authorname
\version{v0.3} % Version number, used in omslag and titelblad, print it with \versionnr

\addresses{} % Your address, this is not currently used anywhere in the template, print it elsewhere with \addressname

\subject{} % Your subject area, this is not currently used anywhere in the template, print it elsewhere with \subjectname
\keywords{} % Keywords for your thesis, this is not currently used anywhere in the template, print it elsewhere with \keywordnames
\university{Hogeschool Rotterdam} % Your university's name and URL, this is used in the title page and abstract, print it elsewhere with \univname
\department{Communicatie, Media, en Informatietechnologie} % Your department's name and URL, this is used in the title page and abstract, print it elsewhere with \deptname
\group{} % Your research group's name and URL, this is used in the title page, print it elsewhere with \groupname
\faculty{} % Your faculty's name and URL, this is used in the title page and abstract, print it elsewhere with \facname

\AtBeginDocument{
\hypersetup{pdftitle=\ttitle} % Set the PDF's title to your title
\hypersetup{pdfauthor=\authorname} % Set the PDF's author to your name
\hypersetup{pdfkeywords=\keywordnames} % Set the PDF's keywords to your keywords
}

\hoofdvraag{Op welke wijze kan Developers.nl de architectuur van hun huidige systemen beter schaal- en onderhoudbaar maken?}
\deelvraagtechnieken{Wat voor technieken voor schaal- en onderhoudbaarheid zijn toepasselijk voor Developers.nl?}
\deelvraaghuidig{Hoe onderhoud- en schaalbaar zijn huidige systemen met betrekking tot de relevante kwaliteitsstandaarden?}
\deelvraagverbetering{Wat voor verbeteringen ten aanzien van schaal- en onderhoudbaarheid kunnen worden toegepast op de huidige systemen?}
\deelvraagimplementatie{Hoe gaan deze verbeteringen geïmplementeerd worden?}
\deelvraagrequirements{Voldoen de verbeteringen aan de vereiste requirements?}
\begin{document}

\renewcommand{\tablename}{Tabel}
\renewcommand{\figurename}{Figuur}
\definecolor{codebg}{rgb}{0.93,0.93,0.93}

% Set texttt background color to grey
\let\oldtexttt\texttt
\renewcommand{\texttt}[1]{
	\colorbox{codebg}{\oldtexttt{#1}}
}

\frontmatter % Use roman page numbering style (i, ii, iii, iv...) for the pre-content pages

\pagestyle{plain} % Default to the plain heading style until the thesis style is called for the body content

%----------------------------
%	OMSLAG
%----------------------------

\begin{titlepage}
	\begin{center}
		
		\vspace*{.06\textheight}
		{\scshape\LARGE \univname\par}\vspace{1.5cm} % University name
		\textsc{\Large Bachelor scriptie}\\[0.5cm] % Thesis type
		
		\HRule \\[0.4cm] % Horizontal line
		{\huge \bfseries \ttitle\par}\vspace{0.4cm} % Thesis title
		\HRule \\[1.5cm] % Horizontal line
		
		\begin{minipage}[t]{0.4\textwidth}
			\begin{flushleft} \large
				\emph{Auteur:}\\\authorname\\0911825
				
			\end{flushleft}
		\end{minipage}
		\begin{minipage}[t]{0.4\textwidth}
			\begin{flushright} \large
				\emph{Begeleiders:} \\
				\techsupname\\
				\skillssupname
			\end{flushright}
		\end{minipage}\\[3cm]
		
		{\large \today}\\[1cm] % Date
		{\Large \texttt{\versionnr}}
		
		\vfill
		
		\includegraphics{HRO} % University/department logo
		
		\vfill
	\end{center}
\end{titlepage}

%----------------------------
%	TITELBLAD
%----------------------------

\begin{titlepage}
	\begin{center}
		{\scshape\LARGE Developers.nl\par}\vspace{0.5cm}
		{\scshape\LARGE \univname\par}\vspace{1.5cm} % University name
		\textsc{\Large Bachelor scriptie}\\[0.5cm] % Thesis type
		
		\HRule \\[0.4cm] % Horizontal line
		{\huge \bfseries \ttitle\par}\vspace{0.4cm} % Thesis title
		\HRule \\[1.5cm] % Horizontal line
		
		\begin{minipage}[t]{0.4\textwidth}
			\begin{flushleft} \large
				\emph{Auteur:}\\\textbf{\authorname}\\
				0911825@hr.nl\\
				k.demunter@developers.nl\\
				06-81019142\\
			\end{flushleft}
		\end{minipage}
		\begin{minipage}[t]{0.4\textwidth}
			\begin{flushright} \large
				\emph{Schoolbegeleiders:}\\
				\textbf{\techsupname}\\
				t.ubert@hr.nl\\
				\textbf{\skillssupname}\\
				j.h.i.lemmens@hr.nl\\[0.3cm]
				
				\emph{Bedrijfsbegeleiders:}\\
				\textbf{Maarten de Boer}\\
				m.deboer@developers.nl\\
				\textbf{Jelle van de Haterd}\\
				j.vandehaterd@developers.nl
			\end{flushright}
		\end{minipage}\\[2cm]
		
		\vfill
		
		\large \textit{Een scriptie ingediend ter voldoening aan de\\ vereiste competenties voor de opleiding \degreename}\\[0.3cm]
		
		\deptname % Research group name and department name
		
		\vfill
		
		{\large \today}\\[0.5cm] % Date
		{\Large \texttt{\versionnr}}\\[1cm]
		
		\includegraphics[height=2cm]{Developers} \hspace{3cm}
		\includegraphics[height=2cm]{HRO}

		\vfill
	\end{center}
\end{titlepage}

%----------------------------
%	ABSTRACT PAGE
%----------------------------

\begin{abstract}
\addchaptertocentry{\abstractname} % Add the abstract to the table of contents
Samenvatting van de scriptie die ik als laatst pas ga schrijven\ldots Lorem ipsum dolor sit amet, consetetur sadipscing elitr, sed diam nonumy eirmod tempor invidunt ut labore et dolore magna aliquyam erat, sed diam voluptua. At vero eos et accusam et justo duo dolores et ea rebum. Stet clita kasd gubergren, no sea takimata sanctus est Lorem ipsum dolor sit amet. Lorem ipsum dolor sit amet, consetetur sadipscing elitr, sed diam nonumy eirmod tempor invidunt ut labore et dolore magna aliquyam erat, sed diam voluptua. At vero eos et accusam et justo duo dolores et ea rebum. Stet clita kasd gubergren, no sea takimata sanctus est Lorem ipsum dolor sit amet. Lorem ipsum dolor sit amet, consetetur sadipscing elitr, sed diam nonumy eirmod tempor invidunt ut labore et dolore magna aliquyam erat, sed diam voluptua. At vero eos et accusam et justo duo dolores et ea rebum. Stet clita kasd gubergren, no sea takimata sanctus est Lorem ipsum dolor sit amet.
\end{abstract}

\begin{voorwoord}
\addchaptertocentry{\voorwoordname}
Bedankt aan iedereen die mij koffie heeft gebracht afgelopen periode\ldots Lorem ipsum dolor sit amet, consetetur sadipscing elitr, sed diam nonumy eirmod tempor invidunt ut labore et dolore magna aliquyam erat, sed diam voluptua. At vero eos et accusam et justo duo dolores et ea rebum. Stet clita kasd gubergren, no sea takimata sanctus est Lorem ipsum dolor sit amet. Lorem ipsum dolor sit amet, consetetur sadipscing elitr, sed diam nonumy eirmod tempor invidunt ut labore et dolore magna aliquyam erat, sed diam voluptua. At vero eos et accusam et justo duo dolores et ea rebum. Stet clita kasd gubergren, no sea takimata sanctus est Lorem ipsum dolor sit amet. 

Lorem ipsum dolor sit amet, consetetur sadipscing elitr, sed diam nonumy eirmod tempor invidunt ut labore et dolore magna aliquyam erat, sed diam voluptua. At vero eos et accusam et justo duo dolores et ea rebum. Stet clita kasd gubergren, no sea takimata sanctus est Lorem ipsum dolor sit amet.
\end{voorwoord}

%------------------------------------------------
%	LIST OF CONTENTS/FIGURES/TABLES PAGES
%------------------------------------------------

\renewcommand{\contentsname}{Inhoud}
\tableofcontents % Prints the main table of contents

\renewcommand{\listfigurename}{Figurenlijst}
\listoffigures % Prints the list of figures

\renewcommand{\listtablename}{Tabellenlijst}
\listoftables % Prints the list of tables

%----------------------------
%	GLOSSARY
%----------------------------
\begin{begrippen}{p{3cm}|p{10cm}} % Include a list of abbreviations (a table of two columns)
	\textbf{Extensibility} & Product extensibility describes how easy it is to expand a product’s feature set \parencite{Extensibility}.\\
	
	\textbf{Kubernetes} & Kubernetes is an open-source system for automating deployment, scaling, and management of containerized applications \parencite{Kubernetes}.\\
	
	\textbf{Docker} & Docker is a tool designed to make it easier to create, deploy, and run applications by using containers \parencite{Docker}.\\
	
	\textbf{Ansible} & Ansible is a radically simple IT automation engine that automates cloud provisioning, configuration management, application deployment, intra-service orchestration, and many other IT needs \parencite{Ansible}.\\
	
	\textbf{Docker Swarm} & Docker Swarm provides native clustering functionality for Docker containers, which lets you turn a group of Docker engines into a single, virtual Docker engine \parencite{DockerSwarm}.\\
	
	\textbf{Serverless Computing} & En application defined as a set of event-triggered
	functions that execute without requiring the user to explicitly manage servers \parencite{ServerlessComputing}.\\
	
	\textbf{Infrastructure as Code} & Infrastructure as code describes the idea of using a high-level programming language to control IT systems \parencite{IaC}.\\
	
	\textbf{Ansible Tower} & Red Hat Ansible Tower helps you scale IT automation, manage complex deployments and speed productivity. Centralize and control your IT infrastructure with a visual dashboard, role-based access control, job scheduling, integrated notifications and graphical inventory management \parencite{AnsibleTower}.\\
	
	\textbf{Chef} & Deploy new code faster and more efficiently. Automate infrastructure and applications \parencite{Chef}.\\
	
	\textbf{Puppet} & Powerful infrastructure automation and delivery \parencite{Puppet}.\\
	
	\textbf{Terraform} &
	Terraform is a tool for building, changing, and versioning infrastructure safely and efficiently. Terraform can manage existing and popular service providers as well as custom in-house solutions \parencite{Terraform}.\\
	
	\textbf{Cloud computing providers} & Cloud computing is a model for enabling ubiquitous, convenient, on-demand network
	access to a shared pool of configurable computing resources (e.g., networks, servers,
	storage, applications, and services) that can be rapidly provisioned and released with
	minimal management effort or service provider interaction \parencite{CloudComputing}.
	
	Er bestaan veel cloud computing providers, waaronder bijvoorbeeld:
	\begin{itemize}
		\item Amazon web services
		\item DigitalOcean
		\item Microsoft Azure
		\item Google Cloud Platform
	\end{itemize}\\
	
	\textbf{12-factor app} & Een methodologie voor het bouwen van Software as a Service (SaaS) applications \parencite{12Factor}. \\
	
	\textbf{The Open Group Architecture Framework} & A generic framework to build different IT architectures frameworks \parencite{TOGAF}.\\
	
	\textbf{4+1 architectural view model} & A model for describing the architecture of software-intensive systems, based on the use of multiple, concurrent views \parencite{4plus1}. \\
\end{begrippen}


%----------------------------------------------------------------------------------------
%	ABBREVIATIONS
%----------------------------------------------------------------------------------------

\begin{abbreviations}{ll} % Include a list of abbreviations (a table of two columns)

\textbf{MI} & \textbf{M}aintainability \textbf{I}ndex\\
\textbf{CI} & \textbf{C}ontinuous \textbf{I}ntegration\\
\textbf{CD} & \textbf{C}ontinuous \textbf{D}eployment\\
\textbf{VU} & \textbf{V}irtual \textbf{U}ser\\
\textbf{EMS} & \textbf{E}mployee \textbf{M}anagement \textbf{S}ystem\\
\textbf{QoS} & \textbf{Q}uality \textbf{o}f \textbf{S}ervice\\
\textbf{AWS} & \textbf{A}mazon \textbf{W}eb \textbf{S}ervices\\
\textbf{GCP} & \textbf{G}oogle \textbf{C}loud \textbf{P}latform\\
\textbf{OPA} & \textbf{O}pen \textbf{P}olicy \textbf{A}gent\\
\textbf{PaC} & \textbf{P}olicies \textbf{a}s \textbf{C}ode\\
\textbf{IaC} & \textbf{I}nfrastructure \textbf{A}s \textbf{C}ode\\
\textbf{K8s} & \textbf{K}ubernete\textbf{s}\\
% \textbf{SaaS} & \textbf{S}oftware \textbf{a}s \textbf{a} \textbf{S}ervice\\
\textbf{TOGAF} & \textbf{T}he \textbf{O}pen \textbf{G}roup \textbf{A}rchitecture \textbf{F}ramework

\end{abbreviations}

%----------------------------------------------------------------------------------------
%	THESIS CONTENT - CHAPTERS
%----------------------------------------------------------------------------------------

\mainmatter % Begin numeric (1,2,3...) page numbering

\pagestyle{thesis} % Return the page headers back to the "thesis" style

% Include the chapters of the thesis as separate files from the Chapters folder
% Uncomment the lines as you write the chapters

\renewcommand{\chaptername}{Hoofdstuk}

\chapter{Inleiding}

\label{Chapter1}

\section{Aanleiding} \label{Aanleiding}
\enquote{Een visitekaartje voor het bedrijf}. Dat is het uitgangspunt van de interne software bij Developers.nl. Niet alleen qua uiterlijk, maar de code, de infrastructuur en de werkmethodes moeten van hoge kwaliteit zijn. Dit heeft te maken met het feit dat de code uit de website en infrastructuur van Developers.nl open-source wordt gemaakt gedurende deze stage. Het open-source maken van de website betekent dat elke potentiële klant en/of nieuwe medewerker de mogelijkheid heeft om te bekijken wat Developers.nl qua kennis in huis heeft. Het is dus van groot belang dat de kwaliteit gewaarborgd wordt, en dat zo veel mogelijk nieuwe en opkomende technieken worden gebruikt. Dit vereist constant onderhoudswerk. Daarnaast heeft Wheeler \parencite{WhyOpenSource} geconcludeerd dat open-source software voordelen heeft als:
\begin{itemize}
	\item Betere beveiliging
	\item Betere betrouwbaarheid 
	\item Betere prestaties
	\item Betere schaalbaarheid
	\item Mindere onderhoudskosten
\end{itemize}

Developers.nl organiseert maandelijks een \enquote{TechNight}. Op deze TechNight ontvangt de website van Developers.nl een piek aantal bezoekers, het is belangrijk dat deze pieken goed worden afgehandeld zonder enige downtime. Dit betekent dat onderzoek op de kwaliteit van de huidige website belangrijk is. Bovendien zijn er meerdere interne systemen dan alleen de website, zoals bijvoorbeeld het Employee Management System (EMS). Het onderhouden van deze systemen vereist veel tijd en moeite. Dit wordt voornamelijk door stagiairs of tijdelijke hulpkrachten uitgevoerd. In dit onderzoek wordt voornamelijk gefocussed op de website. Dit is de applicatie die het meest frequent gebruikt wordt, en dus de meeste aandacht verdiend.

\section{Belang}
De interne systemen zijn op eerste gezicht van de buitenkant vrij eenvoudig. Developers.nl wilt -- om indruk te wekken op potentiële klanten en nieuwe medewerkers -- onder water een applicatie draaien dat \enquote{te complex} is. Maar; omdat de ontwikkelaars óf \enquote{minder ervaren} stagiairs zijn, óf een tijdelijke hulpkracht zijn kost het onderhouden -- vooral met 5 verschillende systemen -- erg veel tijd, moeite, en als gevolg hiervan: geld.

\section{Doelstelling}
Dit onderzoek heeft tot doel het verkrijgen van inzicht over de schaal- en onderhoudbaarheid van de website van Developers.nl, om vervolgens deze twee factoren in de praktijk te verbeteren.

\section{Probleemstelling}
Bij Developers.nl werken veel verschillende ontwikkelaars voor een erg variabele tijd aan de interne projecten. Dit heeft te maken met het feit dat de ontwikkelaars die eraan werken vaak tussen twee opdrachten in zitten. Dit betekent dat het van hoog belang is dat een ontwikkelaar de omgeving snel kan opzetten en op korte termijn een kwalitatieve toevoeging kan leveren die in productie staat. De workflow moet verder geoptimaliseerd worden om Developers.nl dit te beloven.

\section{Hoofd- en Deelvragen}
\subsubsection{Hoofdvraag}
\hoofdvraagname

\subsubsection{Deelvragen}
\begin{itemize}
	\item \deelverwachtingen
	\item \deeltechnieken
	\item \deelhuidig
	\item \deelverbetering
	\item \deelimplementatie
	\item \deelrequirements
\end{itemize}

\section{Methodologie}
Om antwoord te geven op de hoofdvraag \enquote{\hoofdvraagname} is voornamelijk kwalitatief onderzoek uitgevoerd. Vooronderzoek is uitgevoerd door bestaande literatuur te bestuderen om zo een beter beeld te verkrijgen en om een basis te leggen van de belangrijkste begrippen. Alle bronnen zijn handmatig gecontroleerd op kwaliteit en relevantie. Voor het definiëren van functionele schaalbaarheid is veel overlegd met verschillende software-ontwikkelaars, veel feedback gevraagd aan de community, en zijn zowel formele als informele bronnen samengevoegd om zo tot één concrete definitie te komen. 

Om technieken te vinden die behoren bij schaal- en onderhoudbaarheid is deskresearch uitgevoerd door middel van interviews met ontwikkelaars en bestaande onderzoeken te verzamelen. Hierna zijn deze technieken afgebakend tot de meest relevante die bij dit onderzoek horen. Vervolgens is een beschrijvend onderzoek uitgevoerd op de huidige infrastructuur in hoofdstuk \ref{Chapter4}. Hier zijn bestaande kenmerken en elementen van de huidige infrastructuur tegen de gevonden standaarden en technieken uit hoofdstuk \ref{Chapter3} afgewogen. Om verschillende verbeteringen te vinden is net als hoofdstuk \ref{Chapter3} deskresearch uitgevoerd. Dit bevat voornamelijk interviews met senior ontwikkelaars die deze technieken in de praktijk gebruiken. Hierna zijn de gevonden methodes kwalitatief onderbouwd.

Voor de daadwerkelijke implementatie is allereerst exploratief onderzoek gedaan naar de bijbehorende technieken en hun best-practices. Het doel hiervan is vooral om ideeën op te doen naar mogelijke implementaties.

\section{Planning}
In tabel \ref{tab:planning} is de vooraf opgestelde planning te vinden. Het is mogelijk dat hier vanaf is geweken tijdens het daadwerkelijke onderzoek, maar het geeft een ruw beeld van de tijdsverdeling.

\begin{table}[H]
	\caption{Planning}
	\label{tab:planning}
	\centering
	\begin{tabular}{c p{12cm}}
		\toprule
		\textbf{Week} & \textbf{Taak}\\
		\midrule
			1, 2 & Skelet opzet scriptie, inleiding \\
			3, 4 & Theoretisch kader, afbakening \\
			5, 6 & \deeltechnieken \\
			7, 8 & \deelhuidig \\
			9, 10 & \deelverbetering \\
			11, 12 & \deelimplementatie \\
			13 -- 17 & Praktijk implementatie \\
			18 -- 20 & Conclusie en deelvraag \enquote{\deelrequirements} \\
		\bottomrule\\
	\end{tabular}
\end{table}

\section{Leeswijzer}

Vóór het \enquote{echte onderzoek} is eerst in het theoretisch kader (hoofdstuk in \ref{Chapter2}) een literatuuronderzoek uitgevoerd naar definities van de meest belangrijke begrippen: \textbf{Schaalbaarheid}, \textbf{onderhoudbaarheid}, en \textbf{infrastructuur}. Hierdoor is een concrete basis gelegd voor de opvolgende deelvragen. 

In hoofdstuk \ref{Verwachtingen} zijn de wensen en eisen van Developers.nl vastgelegd en requirements opgesteld. Daarna is in \ref{Chapter3} onderzoek gedaan naar de mogelijke technieken, met de deelvraag: \enquote{\deeltechnieken}. Hier zijn standaarden en best-practices besproken om te kunnen bewijzen dat de hoofdvraag daadwerkelijk beantwoord is. 

In het volgende hoofdstuk (\ref{Chapter4}) met deelvraag \enquote{\deelhuidig} zijn deze standaarden afgewogen tegen de huidige infrastructuur. Vervolgens wordt in hoofdstuk \ref{Chapter5} onderzocht welke verbeteringen hier op toe te passen zijn, hierbij hoort de deelvraag \enquote{\deelverbetering}. In hoofdstuk \ref{Chapter6} wordt de daadwerkelijke implementatie van deze verbeteringen besproken door verschillende opties met elkaar af te wegen. Hier wordt antwoord gegeven op de deelvraag \enquote{\deelverbetering}. 

Hierna zijn de geïmplementeerde verbeteringen geëvalueerd in hoofdstuk \ref{Chapter7}, bijbehorende deelvraag \enquote{\deelrequirements}. Nu alle deelvragen zijn beantwoord worden er in hoofdstuk \ref{Chapter8} aanbevelingen toegelicht voor toekomstige verbeteringen. Ten slotte wordt in hoofdstuk \ref{Chapter9} antwoord gegeven op de hoofdvraag \enquote{\hoofdvraagname} door alle deelconclusies samen te binden tot één hoofdconclusie. In hoofdstuk \ref{Reflectie} staat een zelfreflectie over het onderzoek.

\section{Opdrachtgever}

Deze scriptie is geschreven in opdracht van Developers.nl.

\subsection{Core business}
Developers.nl neemt software ontwikkelaars in dienst. De ontwikkelaars die worden aangenomen zullen voornamelijk gespecialiseerd zijn in PHP, Python, Java of front-end. Ze worden uitgezet naar een klant (een extern bedrijf) die naar een ontwikkelaar zoekt. Developers.nl kiest hier voor de beste ontwikkelaar voor de taak en zal deze inzetten bij een klant. De opdrachten van de ontwikkelaars zijn op locatie van de klant en duren voornamelijk langer dan een jaar, maar op uitzondering zijn er ook kortere opdrachten. Zodra de ontwikkelaar klaar is met zijn of haar taak zal Developers.nl zo snel mogelijk een nieuwe opdracht toewijzen \parencite{Stageplan}. Concreet zegt het positioneringsprofiel \parencite{Positioneringsprofiel}: \enquote{Detachering van developers die software applicaties bouwen voor verschillende klanten.}

\subsection{Eigen omgeving}

Tijdens de stageperiode neemt de stagiair een leidende rol aan in een team van 2 part-time studenten, een derdejaars-stagiair, en de tijdelijke hulpkrachten. Developers.nl heeft rond de 60 software ontwikkelaars. Deze zijn voornamelijk op een externe opdracht bij een klant. Elke vrijdag zullen 5 \enquote{kennisambassadeurs} op kantoor zijn. Dit zijn de meest senior ontwikkelaars per team. Deze zijn dan in staat om stagiairs en/of andere medewerkers persoonlijk te helpen. Hoewel ze maar één keer per week op kantoor aanwezig zijn, zijn ze altijd telefonisch bereikbaar of via Slack. Daarnaast kijken de kennisambassadeurs code van de interne systemen inhoudelijk na en geven hier feedback op.

De bedrijfsbegeleider voor deze stage is Maarten de Boer. Dit is de algemene directeur van Developers.nl en is in 2003 afgestudeerd aan de hogeschool Inholland met Strategic marketing. Aangezien Maarten zelf geen technische kennis heeft is er ook een technische begeleider aangewezen: Jelle van de Haterd. Jelle is senior developer, DevOps engineer en kennisambassadeur bij Developers.nl. Hij is in 2006 afgestudeerd op de Hogeschool Rotterdam met als opleiding Grafimediatechnologie \parencite{Afstudeervoorstel}.
\chapter{Theoretisch Kader}
\label{Chapter2}

In dit hoofdstuk worden drie belangrijke begrippen uit de onderzoeksvraag behandeld. Er wordt een literatuuronderzoek gedaan naar de bestaande definities van schaalbaarheid, onderhoudbaarheid en architectuur met betrekking tot software. Hierna wordt het begrip afgebakend tot een concrete definitie waar het onderzoek op terug kan vallen.

\section{Schaalbaarheid}

M. D. Hill heeft in 1990 onderzoek gedaan naar een concrete definitie naar schaalbaarheid \parencite{WhatIsScalability}. In zijn onderzoek concludeert hij het volgende:
\begin{quote}
	\textit{
		I examined aspects of scalability, but did not find a useful, rigorous definition of it. Without such a definition, I assert that calling a system ‘scalable’ is about as useful as calling it ‘modern’. I encourage the technical community to either rigorously define scalability or stop using it to describe systems.
	}
\end{quote}
Ondertussen zijn er meerdere pogingen gedaan om schaalbaarheid te definiëren. zo zijn L. Duboc, D. S. Rosenblum en T. Wicks op deze conclusie ingegaan en hebben een poging gedaan om een framework te creëren voor karakterisering en analyse van software schaalbaarheid \parencite{ScalabilityFramework}. Zij definiëren schaalbaarheid als:
\begin{quote}
	\textit{quality of software systems characterized by the causal impact that scaling aspects of the system environment and design have on certain measured system qualities as these aspects are varied over expected operational ranges}
\end{quote}

% TODO: uitleggen scalability framework?

In een onderzoek over de kenmerken van schaalbaarheid en de impact op prestatie heeft A. B. Bondy \parencite{ScalabilityCharacteristics} schaalbaarheid verdeeld in een aantal verschillende aspecten, waaronder:
\begin{itemize}
	\item \textbf{Structural scalability} (het vermogen van een systeem om uit te breiden in een gekozen dimensie zonder ingrijpende wijzigingen in de architectuur)
	\item \textbf{Load scalability} (het vermogen van een systeem om elegant te presteren naarmate het aangeboden verkeer toeneemt)
	\item \textbf{Space scalability} (het geheugenvereiste groeit niet naar \enquote{ondraaglijke niveaus} naarmate het aantal items toeneemt)
	\item \textbf{Space-time scalability} (het systeem blijft naar verwachtingen functioneren naarmate het aantal items dat het omvat toeneemt)
\end{itemize}
Bondy definieert schaalbaarheid als het vermogen van een systeem om een toenemend aantal elementen, objecten en werk gracieus te verwerken en / of vatbaar te zijn voor uitbreiding.

H. El-Rewini en M. Abd-El-Barr noemen in het boek Advanced computer architecture and parallel processing \parencite{AdvancedArchitecture} ook een aantal \enquote{onconventionele} definities:
\begin{itemize}
	\item \textbf{Size scalability} (Meet de maximale hoeveelheid processors dat een systeem kan accommoderen)
	\item \textbf{Application scalability} (de mogelijkheid om applicatiesoftware te draaien met verbeterde prestaties op een opgeschaalde versie van het systeem)
	\item \textbf{Generation scalability} (de mogelijkheid om op te schalen door het gebruik van de volgende generatie (snelle) componenten)
	\item \textbf{Heterogeneous scalability} (het vermogen van een systeem om op te schalen met behulp van hardware- en softwarecomponenten die door verschillende leveranciers zijn gemaakt)
\end{itemize}

C. B. Weinstock en J. B. Goodenough hebben een algemeen onderzoek uitgevoerd naar schaalbaarheid \parencite{OnSystemScalability}. Zij noemen in hun conclusie dat er voornamelijk twee betekenissen van het woord schaalbaarheid zijn:
\begin{enumerate}
	\item De mogelijkheid om met verhoogde werkdruk om te gaan (zonder extra resources aan een systeem toe te voegen).
	\item De mogelijkheid om met verhoogde werkdruk om te gaan door herhaaldelijk een kosteneffectieve strategie toe te passen om de mogelijkheden van een systeem uit te breiden.
\end{enumerate}
Het valt op dat een concrete definitie van schaalbaarheid alleen duidelijk te definiëren is wanneer het in meerdere verschillende soorten is opgesplitst. Daarom zal in dit onderzoek vanaf dit punt altijd worden gespecificeerd welke soort schaalbaarheid het betreft. In dit onderzoek wordt vooral de focus gelegd op de \enquote{structural scalability} en \enquote{load scalability} uit \parencite{ScalabilityCharacteristics} omdat deze het meest relevant zijn met betrekking tot de probleemstelling. \enquote{application scalability} uit \parencite{AdvancedArchitecture} heeft veel overlapping met de load scalability. Omdat load scalability iets generieker is en de twee definities van Weinstock en Goodenough \parencite{OnSystemScalability} omvat wordt deze geprefereerd boven application scalability. De overgebleven definities zijn minder relevant voor dit onderzoek aangezien ze te maken hebben met hardware, of niet volledig toepasselijk is op de architectuur.

\subsection{Schaalbaarheid controleren}
In het onderzoek van Weinstock en Goodenough \parencite{OnSystemScalability} zijn een aantal manieren genoemd om de schaalbaarheid van een systeem te controleren.

\subsubsection{Knelpunten}
Deze knelpunten hebben vooral te maken de eerste betekenis van Weinstock en Goodenough. Er moet gecontroleerd worden op de toenemende administratieve werkdruk, de \enquote{hard-coded} limieten op capaciteit, de user interface en de complexiteitsgraad van algoritmen.

\subsubsection{Onthullen van schaalbaarheids-aannames}
Dit gaat over het onderzoeken hoe de uitbreiding van een systeem nieuwe problemen kan onthullen. Zodra een systeem zich uitbreidt is er een grotere kans op errors in de systeemconfiguratie, \enquote{zeldzame} errors komen vaker voor, is het belangrijk dat een probleem in het systeem gelokaliseerd blijft, en kan het een stuk complexer en lastiger worden om het systeem te begrijpen.

\subsubsection{Schaalstrategieën}
Hier wordt bekeken wat voor verschillende tekortkomingen een schaalstrategie heeft. Zodra een systeem een lange termijn leeft kan het zo zijn dat het systeem anders wordt gebruikt, hier moet de strategie op voorbereid zijn. Een strategie moet moet niet te afhankelijk zijn van het feit dat gebruikers weten hoe het systeem wordt geïmplementeerd. Ook moet de strategie voorbereid zijn op de vooruitgang van hardware.

\subsubsection{Methoden voor schaalbaarheidsborging}
In het onderzoek vertellen Weinstock en Goodenough dat niet echt mogelijk is om te testen of een systeem schaalbaar is. Wel zijn er methoden om de schaalbaarheid te waarborgen, zoals:
\begin{itemize}
	\item Onderzoek de \enquote{performance curves} en karakteriseer deze met een Big O notatie.
	\item Identificeer mechanismen om knelpunten aan het licht brengen of waar aannames van het schaalbaarheids- ontwerp beginnen te worden geschonden.
	\item Voer een SWOT analyse uit op de schaalbaarheids-strategie.
	\begin{itemize}
		\item \textbf{S}trengths (de soorten groei waar de strategie voor ontworpen is)
		\item \textbf{W}eaknesses (de soorten groei waar de strategie niet voor ontworpen is)
		\item \textbf{O}pportunities (mogelijke veranderingen in werklast of technologie die de strategie goed zou kunnen benutten)
		\item \textbf{T}hreats (mogelijke veranderingen in de werklast of technologie die de strategie in twijfel zouden kunnen trekken)
	\end{itemize}
\end{itemize}

Schalen kan op twee verschillende manieren, namelijk horizontaal en verticaal. Horizontaal wilt zeggen dat er geschaald wordt door meerdere machines toe te voegen, terwijl verticaal schalen betekent dat er  meer rekenkracht (als bijvoorbeeld een betere CPU of meer RAM) wordt toegevoegd aan een machine. Ook is bij het schaalbaar maken van systemen van belang dat het zo min mogelijk ten koste gaat van prestaties en niet meer kost dan nodig is.

\subsection{Functionele schaalbaarheid}
Wat in de literatuur mist over schaalbaarheid is het schrijven van \enquote{schaalbare code}. Deze term wordt regelmatig gebruikt in informele bronnen als blogs, maar is nooit concreet gedefinieerd. Schaalbare code betekent in welke mate bestaande code moet worden aangepast zodra een nieuwe functionaliteit wordt toegevoegd aan het systeem. In dit onderzoek refereren we naar deze definitie als \enquote{functionele schaalbaarheid}. Een aantal informele bronnen gebruiken vaak een definitie in de richting van \enquote{De mogelijkheid om een systeem te verbeteren door nieuwe functionaliteit toe te voegen zonder bestaande activiteiten te verstoren}. Het is echter niet duidelijk waar deze definitie vandaan komt. Deze tak van schaalbaarheid sluit de complexiteit van algoritmes uit, dit valt namelijk meer onder load scalability. Functionele schaalbaarheid is een onderdeel van onderhoudbaarheid (meer over deze definitie in paragraaf \ref{onderhoudbaarheid}).

\section{Onderhoudbaarheid} \label{onderhoudbaarheid}
P. Grubb en A. A. Takang definiëren in hun boek \enquote{Software Maintenance: Concepts And Practice} onderhoudbaarheid als \enquote{The discipline concerned with changes related to a software system after delivery} \parencite{MaintenanceConcepts}. In 1993 heeft IEEE een \enquote{Standard Glossary of Software Engineering Terminology} opgesteld. Deze begrippenlijst definieert onderhoudbaarheid als \enquote{the ease with which a software system or component can be modified to correct faults, improve performance or other attributes, or adapt to a changed environment} \parencite{SENTerminology}. Deze twee definities komen uiteindelijk op hetzelfde neer. Grubb en Takang noemen het in de context van een discipline, terwijl IEEE het als een kwaliteitseigenschap definieert. Ook specificeren Grubb en Takang het feit dat het alleen ná het opleveren gebeurt. In dit onderzoek wordt uitgegaan van de definitie van IEEE.

Grubb en Takang noemen ook een aantal redenen waarom software moet worden onderhouden:
\begin{itemize}
	\item Ondersteuning van verplichte upgrades
	\item Ondersteuning van verzoeken van gebruikers om verbeteringen toe te voegen
	\item Om toekomstige onderhoudswerkzaamheden te vergemakkelijken
\end{itemize}

K.K. Aggarwal et al. noemen in hun onderzoek een aantal factoren die van invloed zijn op onderhoudbaarheid van software \parencite{MaintainabilityMeasure}:
\begin{itemize}
	\item Leesbaarheid van de broncode
	\item Kwaliteit van de documentatie
	\item Begrijpelijkheid van software
\end{itemize}

ISO 25010 \parencite{ISO25010} definieert onderhoudbaarheid als \enquote{The degree of effectiveness and efficiency with which a product or system can be modified to improve it, correct it or adapt it to changes in environment, and in requirements} en verdeeld het in een vijftal kwaliteitseigenschappen.
\begin{itemize}
	\item Modulariteit
	\item Herbruikbaarheid
	\item Analyseerbaarheid
	\item Wijzigbaarheid
	\item Testbaarheid
\end{itemize}

Een bekende manier om onderhoudbaarheid te meten is de zogenaamde Maintainability Index (MI). Hier is echter veel kritiek op \parencite{MaintainabilityLiteratureReview, WhyNoMI, WhyNoMI2}.

\section{Architectuur}
P. Kruchten noemt dat software-architectuur zich bezig houdt met het ontwerp en de implementatie van de structuur op hoog niveau \parencite{4plus1}. Dit is echter een vrij vage definitie, het is niet duidelijk wat \enquote{hoog niveau} precies inhoudt.

S. T. Albin definieert software-architectuur als \enquote{De waarneembare eigenschappen van een softwaresysteem} \parencite{ArtOfArchitecture}. Ook dit is een onduidelijke definitie, het is veel te algemeen.

L. Bass en P. Clements, definiëren de architectuur van software als het volgende \parencite{ArchitectureInPractice}: \enquote{The architecture of a software-intensive system is the structure or structures of the system, which comprise software elements, the externally visible properties of those elements, and the relationships among them}. Gerespecteerde boeken als \parencite{ArchitectureStakeholders, DesigningArchitectures} nemen deze definitie als uitgangspunt. Ook noemen Bass en Clements vier verschillende aspecten die behoren bij software-architectuur:
\begin{itemize}
	\item \textbf{Statische structuur} (interne design-time elementen zoals modules, classes, packages, services, of andere zelfstandige code-eenheden en hun opstelling.)
	\item \textbf{Dynamische structuur} (de runtime-elementen zoals informatie-flows, parallelle of opeenvolgende uitvoering van interne taken, of de invloed die ze hebben op data en hun interacties.)
	\item \textbf{Extern zichtbaar gedrag} (de functionele interacties tussen het systeem en zijn omgeving. Denk aan Informatie-flows in en uit het systeem, of API's.)
	\item \textbf{Kwaliteitseigenschappen} (externe zichtbare, niet-functionele eigenschappen van een systeem zoals prestaties, beveiliging of schaalbaarheid.)
\end{itemize}

ISO/IEC/IEEE 42010:2011 definieert software-architectuur als \enquote{Fundamental concepts or properties of a system in its environment embodied in its elements, relationships, and in the principles of its design and evolution} \parencite{IEEEArchitecture}. The Open Group Architecture Framework (TOGAF) voegt nog een tweede definitie toe aan deze context \parencite{ArchitectureTOGAF}: \enquote{The structure of components, their inter-relationships, and the principles and guidelines governing their design and evolution over time}. TOGAF is gebaseerd op een viertal architectuur-domeinen: business, data, applicatie en technische/infrastructuur architectuur. In dit onderzoek wordt alleen de technische/infrastructuur architectuur gebruikt. Dit domein omvat de IT infrastructuur, middleware, netwerken, communicaties en standaarden. Onder deze definitie passen ook de vier aspecten uit \parencite{ArchitectureInPractice}.

\section{de "Twelve-Factor App"}

A. Wiggins \parencite{12Factor} heeft een methodologie opgezet om moderne, schaalbare en onderhoudbare web-applicaties te bouwen. De methodologie past goed bij de probleemstelling, het minimaliseren van de kosten en tijd die het kost om nieuwe ontwikkelaars aan het project te laten werken en de gemakkelijkheid van het schalen. Ook zorgt het voor structural scalability, draagbaarheid tussen uitvoeringsomgevingen, de mogelijkheid om te deployen op moderne cloud platformen en een minimale divergentie tussen development en productie waardoor CD gemakkelijk wordt om te implementeren.

De methodologie heeft 12 factoren die voor deze eigenschappen zorgen. In tabel \ref{tab:FactorApp} is een gevolg samen met een uitleg bij elke factor geplaatst. Deze factoren zullen worden meegenomen bij het behandelen van de deelvragen.

\begin{table}[h]
	\caption{12-Factor app}
	\label{tab:FactorApp}
	\centering
	\begin{tabular}{c p{3cm} p{3.5cm} p{5.5cm}}
		\toprule
		\textbf{Factor} & \textbf{Naam} & \textbf{Gevolg} & \textbf{Waarom?} \\
		\midrule
		1 & Codebase & Onderhoudbaarheid & Door de frequente deploys kunnen veranderingen snel in productie gezet worden. \\
		2 & Dependencies & Onderhoudbaarheid & gemakkelijk opzetten van project voor nieuwe ontwikkelaars. \\
		3 & Config & Structural scalability & Environment variabelen zijn niet in omgevingen maar per deployment opgezet, zo maakt de hoveelheid omgevingen niet uit. \\
		4 & Backing services & Onderhoudbaarheid & Door backing services als \enquote{attached resources} te behandelen maakt het niet uit welke techniek er wordt gebruikt en zijn deze dus los gekoppeld. \\
		5 & Build, release, run & Onderhoudbaarheid & Het is onmogelijk om veranderingen aan de code tijdens runtime te maken. \\
		6 & Processes & Load scalability & Mede door de shared-nothing architectuur kan het systeem gemakkelijker schalen. \\
		7 & Port binding & \textbf{Niet relevant ?} & \textbf{Niet relevant ?} \\
		8 & Concurrency & Load scalability & Door het \enquote{process model} te gebruiken is het makkelijker om horizontaal en verticaal te schalen. \\
		9 & Disposability & Load scalability & Door processen gemakkelijk te laten stoppen en starten gaat het schalen een stuk sneller.  \\
		10 & Dev/prod parity & Onderhoudbaarheid & Hierdoor kan er een stuk vaker gedeployed worden. \\
		11 & Logs & Onderhoudbaarheid & Door logs naar de \texttt{stdout} te sturen is het gemakkelijker om specifieke fouten te vinden, overzicht te creëren en actief meldingen te versturen naar ontwikkelaars. \\
		12 & Admin processes & Onderhoudbaarheid & Door commands in versiebeheer op te slaan is er een duidelijk overzicht en een geschiedenis van alle \enquote{one-off processes} die gebeuren. \\
		\bottomrule\\
	\end{tabular}
\end{table}

\chapter{Technieken}

\label{Chapter3}

Deze paragraaf gaat over de deelvraag \enquote{\deeltechnieken}.

\section{de "Twelve-Factor App"}

A. Wiggins \parencite{12Factor} heeft een methodologie opgezet om moderne, schaalbare en onderhoudbare web-applicaties te bouwen. De methodologie past goed bij de probleemstelling, het minimaliseren van de kosten en tijd die het kost om nieuwe ontwikkelaars aan het project te laten werken en de gemakkelijkheid van het schalen. Ook zorgt het voor structural scalability, draagbaarheid tussen uitvoeringsomgevingen, de mogelijkheid om te deployen op moderne cloud platformen en een minimale divergentie tussen development en productie waardoor CD gemakkelijk wordt om te implementeren.

De methodologie heeft 12 factoren die voor deze eigenschappen zorgen. Vijf jaar nadat Wiggins de 12 factor app heeft opgesteld is K. Hoffman aan de slag gegaan met een boek genaamd \enquote{Beyond the 12-factor app} \parencite{Beyond12Factor}. Dit boek heeft als doel om de 12 factoren concreter te definiëren en voegt daarnaast nog 3 extra factoren toe om applicaties in de cloud niet alleen te laten functioneren maar ook te laten gedijen. Deze factoren zijn telemetry, security, en het concept \enquote{API first}. In tabel \ref{tab:FactorApp} is een gevolg samen met een uitleg bij elke factor geplaatst. Deze factoren zullen worden meegenomen bij het behandelen van de deelvragen.

\begin{table}[h]
	\caption{12-Factor app}
	\label{tab:FactorApp}
	\centering
	\begin{tabular}{c p{3cm} p{3.5cm} p{5.5cm}}
		\toprule
		\textbf{Factor} & \textbf{Naam} & \textbf{Gevolg} & \textbf{Waarom?} \\
		\midrule		
		1 & One codebase, one application & Onderhoudbaarheid & Door de frequente deploys kunnen veranderingen snel in productie gezet worden. \\
		2 & API first & Structural scalability & Door de API op de eerste rang te zetten van het development proces wordt de mogelijkheid gecreëerd om met elkaars contracten te communiceren zonder interne ontwikkelingsprocessen te verstoren. Zo kunnen veel nieuwe services gemakkelijker worden toegevoegd. \\
		3 & Dependency management & Onderhoudbaarheid & Gemakkelijk opzetten van project voor nieuwe ontwikkelaars. \\
		4 & Design, build, release, and run & Onderhoudbaarheid & Het is onmogelijk om veranderingen aan de code tijdens runtime te maken. \\
		5 & Configuration, credentials, and code & Structural scalability & Environment variabelen zijn niet in omgevingen maar per deployment opgezet, zo maakt de hoeveelheid omgevingen niet uit. \\
		6 & Logs & Onderhoudbaarheid & Door logs naar de \texttt{stdout} te sturen is het gemakkelijker om specifieke fouten te vinden, overzicht te creëren en actief meldingen te versturen naar ontwikkelaars. \\
		7 & Disposability & Load scalability & Door processen gemakkelijk te laten stoppen en starten gaat het schalen een stuk sneller. \\
		8 & Backing services & Onderhoudbaarheid & Door backing services als \enquote{attached resources} te behandelen maakt het niet uit welke techniek er wordt gebruikt en zijn deze dus los gekoppeld. \\
		9 & Environment parity & Onderhoudbaarheid & Hierdoor kan een stuk vaker gedeployed worden. \\
		10 & Administrative processes & Onderhoudbaarheid & Door commands in versiebeheer op te slaan is er een duidelijk overzicht en een geschiedenis van alle \enquote{one-off processes} die gebeuren. \\
		11 & Port binding & \textbf{Begrijp ik nog niet} & \textbf{Vaag} \\
		12 & Stateless processes & Load scalability & Mede door de shared-nothing architectuur kan het systeem gemakkelijker schalen. \\
		13 & Concurrency & Load scalability & Door processen gemakkelijk te laten stoppen en starten gaat het schalen een stuk sneller. \\
		14 & Telemetry & \textbf{kaas} & \textbf{kaas} \\
		15 & Authentication and authorization & \textbf{kaas} & \textbf{kaas} \\
		\bottomrule\\
	\end{tabular}
\end{table}



\section{ISO 25010}

Kaas
\chapter{Huidige situatie}

\label{Chapter4}

Dit hoofdstuk gaat over de deelvraag \enquote{\deelhuidig}. Dit wordt beantwoord

\section{Huidige Architectuur}
De huidige website is een combinatie van een PHP \& Symfony back-end API en Content Management Systeem, samen met een React + next.js front-end. De infrastructuur is momenteel gebouwd op Docker(-compose) + Ansible. Bitbucket pipeline wordt gebruikt voor het Continuous Integration / Deployment. 

In figuur \ref{fig:infra} is een component diagram te vinden van de huidige website structuur. De front-en backend structuur bevat 5 docker containers:
\begin{itemize}
	\item \textbf{PHP-FPM} (back-end)
	\item \textbf{Nginx} (front-en backend)
	\item \textbf{Redis} (back-end)
	\item \textbf{NodeJS} (front-end)
	\item \textbf{PostgreSQL} (back-end)
\end{itemize}

\begin{figure}
	\centering
	\includegraphics[width=13cm]{Figures/Infrastructure}
	\decoRule
	\caption[Infrastructuur]{Infrastructuur website front-en backend \parencite{Documentation}}
	\label{fig:infra}
\end{figure}

PHP-FPM is een FastCGI Process Manager, deze Container serveert de Symfony “FosREST” API en het Content Management Systeem. De NodeJS container serveert een statische Next.js React applicatie en maakt gebruik van Server Side Rendering. Er zit een Nginx reverse proxy in die kiest om een request naar de back-end of de front-end te laten gaan. Redis is een Key-Value Database die gebruikt wordt voor het cachen, en een PostgreSQL container als database. De Bitbucket Pipeline gebruikt Ansible om op de servers de geüpdatete containers te pullen en te starten.

Voor zowel de front- als backend is één monitoring tool genaamd \enquote{Sentry} geïmplementeerd. Sentry creëert een duidelijk overzicht voor alle errors die opkomen in productie.

Ook heeft Developers.nl een \enquote{Employee Management Systeem} (EMS) gebouwd. Deze heeft een soortgelijke structuur aan de website. Het EMS bevat zeer veel gevoelige informatie en het is dus van hoog belang dat deze goed beveiligd is.

\section{Metingen}

Nu de infrastructuur in kaart is gebracht luidt de vraag; hoe schaalbaar is deze infrastructuur eigenlijk? Om dit te beantwoorden worden de verschillende definities van schaalbaarheid individueel behandeld.

\subsection{Structural scalability}
Definitie: Het vermogen van een systeem om uit te breiden in een gekozen dimensie zonder ingrijpende wijzigingen in de architectuur.

Bij structural scalability horen factor 2 \textbf{(API First)} en 5 \textbf{(Configuration, credentials, and code)} van de 15-factor app. 

\subsubsection{API First}
De website van Developers.nl is momenteel in 2 delen gesplitst: de React Front-end en de PHP API als back-end. Deze worden apart ontwikkeld, waardoor dus het principe altijd wordt toegepast. Daarnaast heeft het EMS geen API, en is dus out-of-scope.

\subsubsection{Configuration}
Een test om te bewijzen dat alle configuratie correct uit de code is verwerkt, is of de applicatie op elk gewenst moment open-source kan worden gemaakt zonder geclassificeerde informatie vrij te geven.

Voor de website wordt er gebruik gemaakt van docker-secrets en ansible-vault. Deze combinatie zorgt ervoor dat er nooit wachtwoorden, API sleutels en dergelijke plain-text in versiebeheer komt te staan. Deze secrets worden uiteindelijk in de containers als environment variabelen opgeslagen en uitgelezen door Symfony. In het EMS is deze techniek nog niet gebruikt en staan credentials wél plaintext in de repository.

Om aan factor 5 te voldoen moet de configuratiefiles niet per specifieke omgeving (productie, test, staging) gegroepeerd worden maar moeten juist individueel per deployment geregeld worden. Dit gebeurt in zowel het EMS als de website, de bitbucket pipeline heeft zijn eigen specifieke environment variabelen om te gebruiken en de variabelen in de docker containers worden meegegeven in de algemene docker-compose file die in elke deployment hetzelfde zal zijn.

\subsection{Load scalability}
Definitie: Het vermogen van een systeem om elegant te presteren naarmate het aangeboden verkeer toeneemt. Bij load scalability horen factor 12 \textbf{(stateless processes)}, 13 \textbf{(concurrency)} en 7 \textbf{(disposability)} van de 15-factor app methodologie. 

\subsubsection{Stateless processes}
Factor 12 vereist dat de applicatie als één of meerdere \enquote{stateless processes} moet uitgevoerd worden. Bij de PHP containers worden geüploade bestanden weggeschreven naar een volume, dit zorgt ervoor dat de container niet volledig stateless meer is. Ook zijn databases in docker containers geplaatst, dit is een stateful process aangezien het van belang is dat niet alle data verloren gaat zodra de container stopt.

\subsubsection{Concurrency}
Voor factor 13 is het van belang dat een applicatie horizontaal uit te schalen is. Zolang de applicatie aan factor 7 (Disposability) en 12 (Stateless processes) voldoet zit deze factor goed \parencite{Beyond12Factor}.

\subsubsection{Disposability}
Voor factor 7 moet een applicatie opstarttijd minimaliseren. Zodra de docker images de initiële buildtime voorbij zijn kan de applicatie snel uit en aan worden gezet. \textcolor{red}{TODO: HOE SNEL?} % TODO: Hoe snel?

Ook vereist factor 7 dat processen netjes worden afgesloten zodra ze een \texttt{SIGTERM} ontvangen. Zodra een docker container met \texttt{docker stop <container>} gestopt wordt zal er een SIGTERM worden gestuurd naar de draaiende processen. De vier containers met processen zijn PostgreSQL, PHP-FPM, Nginx en Redis. Deze sluiten allemaal netjes af, de outputs zijn te zien in Bijlage \ref{DockerExits}.

Ook moeten de processen bestendig zijn tegen \enquote{sudden death}. Om dit te simuleren kan \texttt{docker kill <container>} gebruikt worden om een \texttt{SIGKILL} te sturen naar de hoofdprocessen. In bijlage \ref{DockerKills} is te zien dat alle containers na een \texttt{docker kill} zonder problemen weer kunnen opstarten.

\subsubsection{Weinstock \& Goodenough controle}
Om de schaalbaarheid te waarborgen zullen de 3 methoden van Weinstock en Goodenough \parencite{OnSystemScalability} uitgevoerd worden. Performance curves zullen worden gevisualiseerd, knelpunten zullen worden uitgelicht en een SWOT analyse op de schaalbaarheid zal worden uitgevoerd.

Om de performance curves te visualiseren zal een load-test worden uitgevoerd. Er zijn hier meerdere tools voor vergeleken, waaronder:
\begin{itemize}
	\item https://loader.io/
	\item https://gatling.io/
	\item https://k6.io/
	\item http://tsung.erlang-projects.org/
\end{itemize}

De gratis versie van loader.io is niet genoeg voor de wensen van de test, voor gatling.io is Ruby kennis nodig, en voor Tsung worden de tests in XML geschreven wat het lastig maakt om de load op te schalen. Uiteindelijk is gekozen voor K6 omdat zo goed als elke ontwikkelaar genoeg Javascript kennis heeft om deze tool te gebruiken. Ook heeft k6 een eenvoudige manier om de hoeveelheid Virtual Users (VU) geleidelijk te verhogen. Om de uitkomsten te visualiseren is InfluxDB samen met Grafana gebruikt. In bijlage \ref{Bijlagek6} is de implementatie hiervan te vinden.

\textcolor{red}{TODO: PERFORMANCE CURVES} % TODO: Performance curves

Één van de grootste limiterende factoren bij het schalen van de website is de hoeveelheid opslag. Voornamelijk omdat het CMS dubbel functioneert als \enquote{file-server}. Daarnaast bevat de content van de website een grote hoeveelheid foto's en video's, waardoor het opslaggebruik snel kan oplopen. Door het commando \texttt{\$ df -h} is te zien dat 21G -- oftewel 57\% -- van de totale 49G wordt gebruikt.
\begin{minted}[bgcolor=codebg, breaklines]{text}
Filesystem      Size  Used Avail Use% Mounted on
/dev/vda1        49G   27G   21G  57% /
\end{minted}
Bij nader onderzoek is te zien dat de statische folder (waar ook de geüploade bestanden in zitten) maar \texttt{278M} is, dus er is nog veel ruimte voor uitbreiding in dit aspect en zal voor een redelijk lange tijd geen probleem vormen:
\begin{minted}[bgcolor=codebg, breaklines]{text}
root@developers:/etc/developers.nl# du -shc ./static/
278M	./static/
\end{minted}



\subsection{Functional scalability}
Definitie: In welke mate bestaande code moet worden aangepast zodra een nieuwe functionaliteit wordt toegevoegd aan het systeem.

Binnen de scope van de infrastructuur.

\subsection{Onderhoudbaarheid}
Definitie: The degree of effectiveness and efficiency with which a product or system can be modified to improve it, correct it or adapt it to changes in environment, and in requirements.

\section{conclusie}

\chapter{Verbeteringen}

\label{Chapter5}

Deze paragraaf gaat over de deelvraag \enquote{\deelverbetering}
 
\chapter{Implementatie}

\label{Chapter6}

Dit hoofdstuk gaat over de deelvraag \enquote{\deelimplementatie}

\section{Codecov}
Code voor het implementeren is te zien in bijlage \ref{codecov} 
\chapter{Requirements}

\label{Chapter7}

Deze paragraaf gaat over de deelvraag \enquote{\deelrequirements}

\section{Conclusie}

Boter, kaas en eieren.
\chapter{Aanbevelingen}

\label{Chapter8}

\section{Cloud service providers}
Momenteel worden de applicaties binnen Developers.nl gehost op een simpele, traditionele server van TransIP. Een overweging om te maken is of dit niet beter naar een cloud service provider kan worden verplaatst, aangezien dit mogelijk de onderhoudbaarheidslast verminderd. L. Wang \textit{et al.} \parencite{CloudPerspective} definiëren cloud computing als \enquote{A computing Cloud is a set of network enabled services, providing scalable, QoS guaranteed, normally personalized, inexpensive computing infrastructures on demand, which could be accessed in a simple and pervasive way}. Volgens \parencite{CloudPlatformsIntroduction} zijn er drie verschillende categorieën van Cloud computing:
\begin{itemize}
	\item Infrastructure as a Service (IaaS): Een virtueel aangeboden infrastructuur van rekenkracht en/of geheugen \parencite{CloudComputingAdvantages}.
	\item Platform as a Service (PaaS): Een aangeboden platform voor ondersteuning van deployment, ontwikkelen en testen van applicaties \parencite{TransformingCloud}.
	\item Software as a Service (SaaS): Een aangeboden (web)applicatie dat direct gebruikt kan worden \parencite{CloudComputingAdvantages}.
\end{itemize}

Om te overwegen of een cloud provider bij de wensen van Developers.nl past worden de voor-en nadelen op een rijtje gezet:
\subsection{Voordelen Cloud}
De kosten van cloud hosting zijn flexibel, er wordt alleen betaalt voor wat er daadwerkelijk gebruikt wordt, zolang er maar verstandig gebruik van wordt gemaakt. Dit betekend dat het mogelijk erg kostenefficiënt kan zijn voor Developers.nl aangezien er tijdens de maandelijkse \enquote{TechNights} piekmomenten zijn op de website, en er heel weinig verkeer is op het EMS. Een ander voordeel is dat er bijna een ongelimiteerde hoeveelheid opslagruimte beschikbaar is. Aangezien het CMS van de website dubbel functioneert als \enquote{file-server} en er veel foto's worden geüpload is het fijn dat er geen rekening hoeft worden gehouden met de hoeveelheid opslag. Bovendien worden software updates automatisch uitgevoerd, software als K8s kunnen al inbegrepen zijn bij de infrastructuur en het maakt schalen gemakkelijker. Aangezien cloud providers meer middelen voor beveiliging hebben wordt de veiligheid ook een stuk verbeterd.

\subsection{Voordelen traditioneel} 
Ook al is cloud hosting meer kostenefficiënt is het toch mogelijk dat een web host goedkoper uitkomt. Zolang er maar geen hoge piekmomenten zijn in het verkeer. Dit is dus niet van toepassing op Developers.nl aangezien de \enquote{TechNights} of andere evenementen voor piekmomenten zorgen. 

\subsection{Nadelen cloud}
Cloud providers hebben de mogelijkheid voor technische problemen buiten de controle van klanten, waardoor het mogelijk is dat er downtime ontstaat. Ook kan het duurder uitpakken zodra er niet goed word omgegaan met de schaalstrategie of benodigde rekenkracht.

\subsection{Nadelen traditioneel}
De mogelijkheid bestaat dat er meer kosten worden gemaakt dan nodig is. Ook is shared-hosting een risico omdat zodra een andere klant veel rekenkracht opeist de kans ontstaat dat de prestaties dalen.

\subsection{Aanbeveling}
Er zijn veel verschillende cloud providers, waarvan de grootste Amazon Web Services (AWS), Microsoft Azure en Google Cloud Platform (GCP) zijn. Het is verstandig om samen met een migratie naar een cloud provider ook zaken mee te nemen als e-mail beheer in het bedrijf, of de inbegrepen software pakketen als Microsoft Office of Google Drive.

Het grootste nadeel van GCP is dat het marktaandeel vergeleken met Azure of AWS veel minder groot is \parencite{MarketShare}. Hierdoor is het minder interessant voor de interne systemen van Developers.nl, ontwikkelaars die tussen opdrachten zitten willen voor hun persoonlijke ontwikkeling iets leren. De meest nuttige kennis in de ogen van Developers.nl zijn de tools met de meeste marktwaarde. Hierdoor is het niet verstandig om GCP te gebruiken.

Zowel AWS als Azure zijn erg complete producten met veel overeenkomsten. Het grootste voordeel dat Azure met zich mee brengt is het feit dat het erg nauw samen werkt met Microsoft, hierdoor kan de rest van het bedrijf eventueel overstappen van GSuite naar Microsoft Office en op die manier kosten besparen door over een algemeen pakket te beschikken. Ook vervallen de kosten voor BitBucket, Jira, en Slack aangezien deze services ook in AWS of Azure zitten. Het grootste voordeel van AWS is dat het meer flexibiliteit bied aangezien het meer features heeft en iets populairder is.

De beslissende factor (kosten niet meegenomen) is dus of Developers.nl bereid is om over te stappen van GSuite naar Microsoft Office. Verder moet er dieper onderzoek worden uitgevoerd over de kosten van het overstappen naar AWS of GSuite.

\section{Serverless computing}
Serverless computing is een \enquote{application defined as a set of event-triggered functions that execute without requiring the user to explicitly manage servers} \parencite{ServerlessComputing}. In verband met de lage hoeveelheid verkeer op het EMS is \enquote{serverless computing} een goede oplossing om kosten te besparen.

Aangezien de website veel verkeer heeft is het niet verstandig om gebruik te maken van serverless computing. Het nut van serverless computing valt hierdoor weg, dit betekent dat het een stuk duurder kan uitpakken dan conventionele methodes.

\section{Container Orchestration}
Container Orchestration met Docker Swarm zou het mogelijk maken om de website op te schalen. Er is bewezen dat de website schaalbaar is, dus er zijn geen implicaties voorzien bij het implementeren. Wel moet er rekening gehouden worden met Traefik, en de integratie hiervan met Docker Swarm.

\section{Eén generieke infrastructuur}
Het maken van een enkele ansible infrastructuur zou het concept \enquote{one codebase, one application} van de 15-factor app verbeteren. Ook zou dit onderhoudbaarheid verbeteren omdat nieuwe applicaties gelijk een infrastructuur op hoog niveau tot beschikking hebben. Daarnaast is het op deze manier eenvoudiger om nog een hoger level van abstractie toe te voegen met Terraform, om de machine in te richten.

\section{Policies}
Er moet onderzoek gedaan worden naar build- en run-time policies. Met Open Policy Agent is veel meer mogelijk dan wat er in dit onderzoek is geïmplementeerd. Bijvoorbeeld policies die evalueren of er geen onveilige poorten worden geopend.

Verder is het belangrijk dat er policies in plaats komen die checken of een container als de \texttt{root} user draait. Nu is het zo dat PHP-FPM én Nginx als root draaien, terwijl dit niet nodig is\footnote{Meer informatie is te vinden in het volgende artikel:\\ https://engineering.bitnami.com/articles/why-non-root-containers-are-important-for-security.html}.
\chapter{Conclusie}

\label{Chapter9}

In dit hoofdstuk wordt antwoord gegeven op de hoofdvraag: \enquote{\hoofdvraagname}.

\section{Verwachtingen}
Developers.nl wilt onderhoudbaarheid bereiken door kwaliteitsstandaarden af te dwingen. Ook wilt Developers.nl twee soorten schaalbaarheid. Eén in de vorm van het deployen van verschillende branches in aparte omgevingen, en één in de vorm van horizontaal schalen om meer verkeer aan te kunnen. Deze twee soorten kunnen niet gerealiseerd worden zonder de onderhoudbaarheid te waarborgen. Er zijn vijf must-have requirements:

\begin{itemize}
	\item De oplossing moet méér dan twee unieke instanties van de website naast elkaar kunnen draaien.
	\item De oplossing moet moet unieke instanties van de website automatisch kunnen aanmaken.
	\item De oplossing moet een methode bevatten om kwaliteit van nieuwe toevoegingen aan de infrastructuur automatisch te waarborgen.
	\item Minimaal één monitoring tool voor het monitoren van performance.
	\item De oplossing moet voldoen aan één of meerdere kwaliteitsstandaarden.
\end{itemize}

\section{Technieken}
De 12-Factor App is een methodologie die twaalf best-practices samenvoegt om moderne, schaal- en onderhoudbare web-applicaties te bouwen. Het boek \enquote{Beyond the 12-factor app} \parencite{Beyond12Factor} is hierop verder gegaan door nog een drietal factoren toe te voegen. Door een applicatie te evalueren op deze vijftien factoren, samen met de definitie van ISO-25010 \parencite{ISO25010} is te beoordelen of deze schaal- en onderhoudbaar is. Om de schaalbaarheid van een systeem te waarborgen zijn  de methoden van Weinstock en Goodenough \parencite{OnSystemScalability} een geschikte manier.

\section{Huidige situatie}
Om de deelvraag \enquote{\deelhuidig} te beantwoorden zijn door dit onderzoek meerdere punten van verbetering gevonden. Een verbeterpunt in de functionele schaalbaarheid is de hoeveelheid opslag van de server. Deze kan snel vol raken door ongebruikte Docker volumes en images die ontstaan bij een deployment. 

Ook wordt factor 1 (One codebase, one application) van de 15-Factor App niet volledig opgevolgd. De infrastructuur wordt op meerdere plekken opgebouwd en zou netter staan in een aparte codebase. De applicaties voldoen aan factor 13 (Concurrency) maar er wordt nog geen gebruik van gemaakt. 

Factor 11 (Port binding) is voor PHP niet verstandig aangezien PHP juist ontworpen is om een webserver ervoor te hebben, dit verslechtert dus de onderhoudbaarheid. Er zijn voor zowel het EMS als de website geen monitoring tools in gebruik voor performance, dat betekend dat factor 14 (Telemetry) beter kan. Er is geen concrete manier om tests uit te voeren of aan testcriteria is voldaan. Hierdoor is de Testbaarheid van de systemen minimaal.

\section{Verbeteringen}
Om de kenmerken modulariteit en herbruikbaarheid van ISO 25010 te verbeteren kan er een centrale infrastructuur IaC repository gemaakt worden met Ansible. Om de kenmerken analyseerbaarheid, testbaarheid en wijzigbaarheid te verbeteren kunnen policies worden afgedwongen door middel van PaC. Voor het waarborgen van testbaarheid is Codecov een geschikte tool. Om onderhoudbaarheid te verbeteren en schaalbaarheid te bewijzen is het concept van feature-environments erg geschikt. Developers.nl ziet dit concept graag in de praktijk, dit heeft dan ook de hoogste prioriteit als uitkomst van dit onderzoek.

\section{Implementatie}
Er zijn vijf verbeteringen geïmplementeerd: Feature-environments, Codecov, Opschonen van Docker images, Open Policy Agent, en Grafana.

Codecov is in de back-end van de website geïmplementeerd om coverage te visualiseren op Pull-Requests, het genereren hiervan gebeurt in de \texttt{php7-fpm} Dockerfile in een test-buildstage.

Docker images worden opgeschoond door middel van de Ansible \texttt{docker\_prune} module. Deze ruimt images ouder dan 4 uur op bij elke deployment.

De implementatie van Feature-environments heeft de basis gelegd voor een aantal opvolgende implementaties. Het heeft de infrastructuur, pipeline, en workflow veranderd door Traefik -- een reverse proxy -- te implementeren die er voor zorgt dat meerdere individuele omgevingen naast elkaar kunnen draaien, die te bereiken zijn door subdomeinen. Ook beveiligt het de Docker Socket door de TCP connectie met TLS te encrypten. 

De implementatie van Policy-as-Code gaat verder op de beveiliging van de Docker Socket in en gebruikt Open Policy Agent om de users van docker commando's te limiteren. Ook staat er een opzet voor het evalueren van build-time policies.

Verder wordt Traefik gebruikt om Grafana -- een monitoring dashboard -- te exposen op een subdomein.


\section{Requirements}
Must-requirements:
\begin{itemize}
	\item De oplossing moet méér dan twee unieke instanties van de website naast elkaar kunnen draaien.
    \begin{itemize}
        \item Behaald.
    \end{itemize}

	\item De oplossing moet moet unieke instanties van de website automatisch kunnen aanmaken.
    \begin{itemize}
        \item Behaald tot op zekere hoogte, de instanties kunnen nog niet automatisch worden verwijderd. Hier is een aanbeveling over gemaakt.
    \end{itemize}

	\item De oplossing moet een methode bevatten om kwaliteit van nieuwe toevoegingen aan de infrastructuur automatisch te waarborgen.
    \begin{itemize}
        \item Behaald. Er is een aanbeveling gemaakt om hier verder op in te gaan.
    \end{itemize}

	\item Minimaal één monitoring tool voor het monitoren van performance.
    \begin{itemize}
        \item Niet volledig behaald. Grafana is geïmplementeerd om data te visualiseren, maar er is nog geen data. 
    \end{itemize}

	\item De oplossing moet voldoen aan één of meerdere kwaliteitsstandaarden.
    \begin{itemize}
        \item Behaald, maar er zijn nog verbeteringen. Hier is een aanbeveling over gemaakt.
    \end{itemize}
\end{itemize}

Over de Should, Could en Won't requirements zijn aanbevelingen gemaakt in hoofdstuk \ref{Chapter8}.

\section{Hoofdvraag}

Developers.nl kan schaal- en onderhoudbaarheid verbeteren door middel van de volgende technieken: 
\begin{itemize}
    \item Feature-environments
    \item Codecov
    \item opschonen van Docker images
    \item Open Policy Agent
    \item Grafana en Prometheus
    \item AWS of Azure
    \item Docker Swarm
    \item Eén generieke infrastructuur repository
\end{itemize}


%----------------------------------------------------------------------------------------
%	THESIS CONTENT - APPENDICES
%----------------------------------------------------------------------------------------

\appendix % Cue to tell LaTeX that the following "chapters" are Appendices

% Include the appendices of the thesis as separate files from the Appendices folder
% Uncomment the lines as you write the Appendices

\renewcommand{\appendixname}{Bijlage}
% Appendix A

\chapter{Code} % Main appendix title

\label{BijlageCode} 

\section{Docker-compose opstelling voor k6, InfluxDB \& Grafana}

\label{Bijlagek6}

Om de loadtest met k6, influxDB en grafana op te stellen heeft Loadimpact een docker-compose opstelling gemaakt. Na wat onderzoek is het opgevallen dat deze opstelling erg verouderd is. Daarom is ervoor gekozen om een eigen opstelling te maken:
\begin{minted}[linenos=true, bgcolor=codebg, breaklines]{yaml}
version: '3.4'

networks:
  k6:
  grafana:

services:
  influxdb:
    image: influxdb:1.5.4
    networks:
    - k6
    - grafana
    ports:
      - "8086:8086"
    environment:
      - INFLUXDB_DB=k6
    
  grafana:
    image: grafana/grafana:6.4.1
    networks:
      - grafana
    ports:
      - "3000:3000"
    environment:
      - GF_AUTH_ANONYMOUS_ORG_ROLE=Admin
      - GF_AUTH_ANONYMOUS_ENABLED=true
      - GF_AUTH_BASIC_ENABLED=false
    volumes:
      - ./grafana/datasource.yml:/etc/grafana/provisioning/datasources /datasource.yml
  
  k6:
    image: loadimpact/k6:0.25.1
    networks:
      - k6
    ports:
      - "6565:6565"
    environment:
      - K6_OUT=influxdb=http://influxdb:8086/k6
    volumes:
      - ../k6:/k6
\end{minted}

Hiervoor is een Pull-Request gemaakt naar loadimpact/k6 om dit te verbeteren. \texttt{https://github.com/loadimpact/k6/pull/1183} samen met de issue\\ \texttt{https://github.com/loadimpact/k6/issues/1182}. Hierin is te lezen wat precies de veranderingen waren. De maintainers van k6 waren blij met de verandering en hebben deze geaccepteerd en gemerged naar master. De loadtest is geschreven in javascript met de volgende code:
\begin{minted}[linenos=true, bgcolor=codebg, breaklines]{javascript}
import http from "k6/http";
import { sleep, check } from "k6";

export let options = {
  stages: [
    { duration: "10s", target: 20 },
    { duration: "10s", target: 40 },
    { duration: "10s", target: 60 },
    { duration: "10s", target: 80 },
    { duration: "10s", target: 100 },
    { duration: "10s", target: 120 },
  ]
};

export default function() {
  check(http.get("https://test.developers.nl/"), {
    "is status 200": (r) => r.status === 200
  });
  sleep(1);
};
\end{minted}

\section{k6 load test resultaten}

\label{bijlageloadtest}

In figuur \ref{fig:loadtestcli} is de CLI output van de loadtest te vinden. In figuur \ref{fig:loadtestvus} zijn de oplopende hoeveelheid VUs uitgebeeld in Grafana, en in figuur \ref{fig:loadtestresultaten} is in Grafana te zien hoe lang de requests duren.
\begin{figure}[H]
	\centering
	\includegraphics[width=13cm]{Figures/loadtest}
	\decoRule
	\caption[k6 loadtest CLI resultaten]{k6 loadtest CLI resultaten}
	\label{fig:loadtestcli}
\end{figure}

\begin{figure}[H]
	\centering
	\includegraphics[width=13cm]{Figures/loadtestusers}
	\decoRule
	\caption[k6 loadtest hoeveelheid VUs en requests]{k6 loadtest hoeveelheid VUs en requests}
	\label{fig:loadtestvus}
\end{figure}

\begin{figure}[H]
	\centering
	\includegraphics[width=13cm]{Figures/loadtestrequests}
	\decoRule
	\caption[k6 loadtest resultaten]{k6 loadtest resultaten}
	\label{fig:loadtestresultaten}
\end{figure}


\section{Docker container exits}

\label{DockerExits}

\subsubsection{PostgreSQL}
\begin{minted}[linenos=true, bgcolor=codebg, breaklines]{text}
LOG: received smart shutdown request
LOG: background worker "logical replication launcher" (PID 43) exited with exit code 1
LOG: shutting down
LOG: database system is shut down
\end{minted}

\subsubsection{PHP-FPM}
\begin{minted}[linenos=true, bgcolor=codebg, breaklines]{text}
NOTICE: Terminating ...
NOTICE: exiting, bye-bye!
\end{minted}

\subsubsection{Redis}
\begin{minted}[linenos=true, bgcolor=codebg, breaklines]{text}
1:signal-handler (1570781278) Received SIGTERM scheduling shutdown...
# User requested shutdown...
* Saving the final RDB snapshot before exiting.
* DB saved on disk
* Removing the pid file.
# Redis is now ready to exit, bye bye...
\end{minted}

\subsubsection{Nginx}
\begin{minted}[linenos=true, bgcolor=codebg, breaklines]{text}
[notice] 1#1: signal 15 (SIGTERM) received from 56, exiting
[notice] 48#48: exiting
[notice] 47#47: exiting
[notice] 47#47: exit
[notice] 1#1: signal 14 (SIGALRM) received
[notice] 1#1: signal 17 (SIGCHLD) received from 48
[notice] 1#1: cache manager process 48 exited with code 0
[notice] 1#1: worker process 47 exited with code 0
[notice] 1#1: exit
\end{minted}

\section{Docker container kill \& restarts}

\label{DockerKills}

\begin{minted}[linenos=true, bgcolor=codebg, breaklines]{text}
$ docker ps -q
d0829783af18
f72e9967771b
01dd48ff5a59
fab794731d47
ca510c065d11
3ee85578efb5

$ docker kill $(docker ps -q) 

$ docker ps
d0829783af18
f72e9967771b
01dd48ff5a59
fab794731d47
ca510c065d11
3ee85578efb5

$ docker ps -q

$ docker start $(docker ps -aq)
d0829783af18
f72e9967771b
01dd48ff5a59
fab794731d47
d68d7ab9809c
ca510c065d11
3ee85578efb5
e1866ab6c1af

$ docker ps -q
d0829783af18
f72e9967771b
01dd48ff5a59
fab794731d47
ca510c065d11
3ee85578efb5
\end{minted}

\section{Codecov implementatie} \label{codecov}
Om codecov te implementeren in de website is het volgende gebeurd:\\
De README.md is bijgewerkt:
\begin{minted}[linenos=true, bgcolor=codebg, breaklines]{text}
## Tests

We enforce that code coverage stays acceptable using codecov:

[![codecov](https://codecov.io/bb/developers_nl/developers.nl/branch/ma
ster/graph/badge.svg?token=DzAv79t9Gd)](https://codecov.io/bb/developer
s_nl/developers.nl)
\end{minted}

Bitbucket en codecoverage environment variabelen moesten worden doorgegeven door build arguments. Het builden van de Docker images gebeurd met Ansible:
\begin{minted}[linenos=true, bgcolor=codebg, breaklines]{yaml}
docker_images:
  - dockerfile: docker/php7-fpm/Dockerfile
  path: ../
  name: developersnl/website-php-fpm
  buildargs:
    GROUP_ID: 9000
    USER_ID: 9000
    BITBUCKET_BRANCH: "{{ lookup('env','BITBUCKET_BRANCH') }}"
    BITBUCKET_COMMIT: "{{ lookup('env', 'BITBUCKET_COMMIT') }}"
    BITBUCKET_BUILD_NUMBER: "{{ lookup('env','BITBUCKET_BUILD_NUMBER') }}"
    BITBUCKET_REPO_OWNER: "{{ lookup('env','BITBUCKET_REPO_OWNER') }}"
    BITBUCKET_REPO_SLUG: "{{ lookup('env','BITBUCKET_REPO_SLUG') }}"
    BITBUCKET_PR_ID: "{{ lookup('env','BITBUCKET_PR_ID') }}"
    CODECOV_TOKEN: "{{ lookup('env','CODECOV_TOKEN') }}"
    CI: "{{ lookup('env','CI') }}"
\end{minted}

In de php7-fpm dockerfile zijn de build args omgezet naar environment variablen, een aantal apk packages toegevoegd en is het codecov script toegevoegd:
\begin{minted}[linenos=true, bgcolor=codebg, breaklines]{bash}
FROM application AS test

ENV SYMFONY_PHPUNIT_VERSION 8.0.0

ARG BITBUCKET_BRANCH
ARG BITBUCKET_BUILD_NUMBER
ARG BITBUCKET_REPO_OWNER
ARG BITBUCKET_REPO_SLUG
ARG BITBUCKET_PR_ID
ARG CODECOV_TOKEN
ARG CI
ARG BITBUCKET_COMMIT

ENV BITBUCKET_BRANCH=$BITBUCKET_BRANCH
ENV BITBUCKET_BUILD_NUMBER=$BITBUCKET_BUILD_NUMBER
ENV BITBUCKET_REPO_OWNER=$BITBUCKET_REPO_OWNER
ENV BITBUCKET_REPO_SLUG=$BITBUCKET_REPO_SLUG
ENV BITBUCKET_PR_ID=$BITBUCKET_PR_ID
ENV CODECOV_TOKEN=$CODECOV_TOKEN
ENV CI=$CI

# TODO: Cange VCS_COMMIT_ID to BITBUCKET_COMMIT when https://github.com/codecov/codecov-bash/pull/225 is deployed
ENV VCS_COMMIT_ID=$BITBUCKET_COMMIT

COPY --from=composer:1.9.0 /usr/bin/composer /usr/bin/composer

RUN apk add \
    php7-pdo_sqlite \
    php7-sqlite3 \
    php7-phar \
    php7-pear \
    php7-dev \
    redis \
    curl \
    bash \
    git \
    mercurial \
    findutils \
    g++ \
    make \
 && . /bin/pcov.sh \
 && redis-server --daemonize yes --requirepass test \
 && composer install -d /app/src --optimize-autoloader --no-interaction --no-suggest --no-scripts \
 && chmod u+x,g+x /app/src/bin/phpunit \
 && /app/src/bin/phpunit --configuration /app/src/phpunit.xml --coverage-clover=coverage.xml \
 && curl -s https://codecov.io/bash | bash -s - -X coveragepy
\end{minted}

Er is een script geschreven om pcov te installeren zodat dit kan hergebruikt worden zowel in de `develop.sh` entrypoint als in de test-stage van de dockerfile.
\begin{minted}[linenos=true, bgcolor=codebg, breaklines]{bash}
#!/bin/sh

# Add PHP Coverage ini configuration
echo "- Enabling pcov"
cat <<-EOF > /etc/php7/conf.d/pcov.ini
extension=pcov
pcov.enable=1
EOF

echo "- Installing pcov"
if ! pecl list | grep pcov >/dev/null 2>&1;
then
    pecl install pcov ||
    {
        echo "Could not pecl install pcov" >&2;
        exit 1;
    }
fi
\end{minted}

Om BitBucket een betere ondersteuning te geven met codecov is hier ook een Pull-Request naar codecov-bash gemaakt. Deze is te zien op:\\ \texttt{https://github.com/codecov/codecov-bash/pull/225}. De maintainers van codecov waren tevreden met deze verbeteringen en hebben de Pull-Request geaccepteerd en gemerged.
% Appendix B

\chapter{Tabellen} % Main appendix title

\label{BijlageTabellen} 

\section{Beyond the 12-factor app}

\label{TabelFactors}

In deze tabel worden de 15 factoren behandeld.

\begin{longtable}[c]{c p{3cm} p{3.5cm} p{5.5cm}}
	\toprule
	\textbf{Factor} & \textbf{Naam} & \textbf{Gevolg} & \textbf{Waarom?} \\
	\midrule		
	1 & One codebase, one application & Onderhoudbaarheid & Door de frequente deploys kunnen veranderingen snel in productie gezet worden. \\
	2 & API first & Structural scalability & Door de API op de eerste rang te zetten van het development proces wordt de mogelijkheid gecreëerd om met elkaars contracten te communiceren zonder interne ontwikkelingsprocessen te verstoren. Zo kunnen veel nieuwe services gemakkelijker worden toegevoegd. \\
	3 & Dependency management & Onderhoudbaarheid & Gemakkelijk opzetten van project voor nieuwe ontwikkelaars. \\
	4 & Design, build, release, and run & Onderhoudbaarheid & Het is onmogelijk om veranderingen aan de code tijdens runtime te maken. \\
	5 & Configuration, credentials, and code & Structural scalability & Environment variabelen zijn niet in omgevingen maar per deployment opgezet, zo maakt de hoeveelheid omgevingen niet uit. \\
	6 & Logs & Onderhoudbaarheid & Door logs naar de \texttt{stdout} te sturen is het gemakkelijker om specifieke fouten te vinden, overzicht te creëren en actief meldingen te versturen naar ontwikkelaars. \\
	7 & Disposability & Load scalability & Door processen gemakkelijk te laten stoppen en starten gaat het schalen een stuk sneller. \\
	8 & Backing services & Onderhoudbaarheid & Door backing services als \enquote{attached resources} te behandelen maakt het niet uit welke techniek er wordt gebruikt en zijn deze dus los gekoppeld. \\
	9 & Environment parity & Onderhoudbaarheid & Hierdoor kan een stuk vaker gedeployed worden. \\
	10 & Administrative processes & Onderhoudbaarheid & Door commands in versiebeheer op te slaan is er een duidelijk overzicht en een geschiedenis van alle \enquote{one-off processes} die gebeuren. \\
	11 & Port binding & \textbf{Begrijp ik nog niet} & \textbf{Vaag} \\
	12 & Stateless processes & Load scalability & Mede door de shared-nothing architectuur kan het systeem gemakkelijker schalen. \\
	13 & Concurrency & Load scalability & Door processen gemakkelijk te laten stoppen en starten gaat het schalen een stuk sneller. \\
	14 & Telemetry & \textbf{kaas} & \textbf{kaas} \\
	15 & Authentication and authorization & \textbf{kaas} & \textbf{kaas} \\
	\bottomrule\\
\end{longtable}

\label{Bijlagek6}
%% Appendix B

\chapter{Gesprekken} % Main appendix title

\label{Feedback} 

\section{Requirements}

\label{FeedbackRequirements}

Dit zijn de gemaakte aantekeningen tijdens discussies over de requirements met Jelle:
\begin{itemize}
	\item schaalbaarheid: meerdere omgevingen (feature branches)
	\item Merge train -> automatische merges en deploys
	\item Pulumi
	\item Generieke boilerplate voor een CI/CD Pipeline
	\item segregation of duties
	\item docker swarm voor performance curves te laten zien
	\item Advies over deployment targets
	\item testbaarheid -> static code analysis -> integratie tests ->
	\item Hoeveelheid coverage -> pipeline 
	\item Probleemstelling \& requirements
	\item Functional scalability
	\item Wat te monitoren?
	\item Product owner validatie \& automatisch testen apart
	\item Validaties zo veel mogelijk automatisch (policies, segregation of duties)
	\item Kwaliteit waarborgen -> concreter
	\item Functional scalability -> Extensibility
	\item Extensibility functioneel? niet functioneel? Vragen stellen feedback online!
	\item Monitoring: Metrics als ruimte -> cpu -> memory uiteindelijk verkeer, etc. promethius
	\item terraform (firewalls, netwerken) voor alles tot aan de VM en ansible om de vm af te configureren
	\item TransIP Terraform API
	\item Nexpertise Terraform
\end{itemize}

\section{Feature environments}
\label{FeedbackFeatureEnvironments}

Slack conversatie met Jelle:
\begin{minted}[bgcolor=codebg, breaklines, breaksymbolleft=]{text}
Hey ik heb denk ik iets gevonden wat mij wel een leuke oplossing lijkt:
https://github.com/jwilder/nginx-proxy
Het luistert naar je docker run commands om daaruit environment variabelen te halen; waaronder VIRTUAL_HOST , waardoor het dus iets als VIRTUAL_HOST=${BITBUCKET_BRANCH}.${HOSTNAME} kan worden. Wat vind jij hiervan? Een mogelijke oplossing? Of tenminste een deel hiervan.... zodat er niet een volledige abstractielaag op zit
\end{minted}
\\Reactie van Jelle:
\begin{minted}[bgcolor=codebg, breaklines, breaksymbolleft=]{text}
Zou idd een oplossing zijn, kijk anders ook ffe naar traefik.io
\end{minted}
\\Conversatie over het beveiligen van de Docker Socket:
\begin{minted}[bgcolor=codebg, breaklines, breaksymbolleft=]{text}
Kaj: 
Oke dit lijkt mij wel een probleem:
https://github.com/containous/traefik/issues/4174
Zowel de nginx-proxy als traefik hebben dit.. is dit: https://github.com/Tecnativa/docker-socket-proxy écht de beste oplossing hiervoor, of heb jij toevallig nog een geniale ingeving?

Jelle:
Zou ik me even in moeten verdiepen, met de oplossing die ik met nginx aan het rommelen was gebruikte ik docker labels en docker inspect

Kaj:
Hmm oke oke
Misschien is dat ook wel een goeie, want nginx moet er toch inblijven aangezien traefik geen fastCGI support heeft

Jelle:
Ja maar das dan achter traefik als load balancer zegmaar

Kaj:
Oh.. dus dan heb je traefik statisch geconfigureerd?

Jelle:
Bekijk de docker-compose file: https://medium.com/@luiscoutinh/reverse-proxy-com-docker-traefik-nginx- php-mysql-mosquitto-phpmyadmin-basic-34c95b690f5c

Kaj:
Ja precies, dat is ongeveer hoe het wordt aangeraden. Maar ook die oplossing exposed de docker socket

Jelle:
ja idd dat is een probleem, iig voor productie
Expose the Docker socket over TCP, instead of the default Unix socket file

Kaj:
Dat is eigenlijk het enige waar ik tegenaan loop, want een oplossing met traefik lijkt mij bijna precies wat ik zoek

Jelle:
dat kan wel, via certificaten beveiligen:
https://docs.traefik.io/providers/docker/#docker-api-access

\end{minted}
\section{(niet-)functionele schaalbaarheid}

\label{feedbackschaalbaarheid}

Na het promoten van de geschreven blog\footnote{https://developers.nl/blog/69/Defining-software-scalability-using-requirements} is dit de meest populaire blog van Developers.nl in 2019 geworden. Dit heeft het volgende feedbackpuntje opgeleverd: \enquote{Scalability is a two way thing, so adding and removing should be in the definition (and thought lines)}. Waar ik het volledig mee eens ben, en zal verbeteren in de toekomst. Ook is er een leuke discussie uit gekomen. Marlon Etheredge, MSc vroeg:
\begin{minted}[bgcolor=codebg, breaklines, breaksymbolleft=]{text}
Hi Kaj,

Mijn vraag behoeft enige introductie.

Ik ben werkzaam in een deelgebied van de informatica/software-engineering waarbij performantie zeer belangrijk is, computer graphics. Onze implementaties dienen zo snel als mogelijk antwoorden te geven op soms complexe berekeningen, doorgaans in (minder dan) millisecondes.

Schaalbaarheid in mijn context staat dan ook voor twee dingen; ten eerste gaat het om het niet schenden van tijdsgrenzen waar wij mee te maken hebben (e.g. één frame dient in 1/50 seconde klaar te zijn) onafhankelijk van de hoeveelheid data die verwerkt moet worden. Ten tweede staat schaalbaarheid voor implementaties die rekbaar zijn op basis van veranderende eisen die aan een systeem worden gesteld.

In deze context gaat het dan niet zo zeer om een veranderend systeem, waarbij bijvoorbeeld functionaliteit wordt toegevoegd
("... must be modified as soon ..." in je eerste definitie), of het systeem verbeterd wordt ("... is able to be improved ..." in je tweede definitie), maar eerder om een kwaliteitskenmerk van een systeem in ogenschouw nemend welke eisen mogelijk in de toekomst aan dit systeem gesteld zullen worden en de hoeveelheid energie die het zal kosten om het systeem te laten aansluiten op deze eisen. In die zin denk ik overigens ook dat dit een interessant onderwerp is, aangezien het ontwerpen van dergelijke systemen fundamenteel is aan de informatica.

Mijn concrete vraag aan jou is als volgt; je schrijft:

"Instead of adding more of the same requirement, non-functional requirements like security or usability are always able to be improved. Therefore, scaling a non-functional requirement is the same as improving it. Setting clear requirements helps proving your solution is scalable non-functionally."

Is verandering (bijvoorbeeld in de vorm van verbetering) noodzakelijk voor schaalbaarheid, of is het mogelijk schaalbaarheid te toetsen los van verandering?
\end{minted}
\\Mijn antwoord:
\begin{minted}[bgcolor=codebg,breaklines, breaksymbolleft=]{text}
Schaalbaarheid in jouw context sluit goed aan op mijn twee definities: Je noemt "het niet schenden van tijdsgrenzen ... onafhankelijk van de hoeveelheid data"; in dit geval zijn de berekeningen een functionele requirement, en deze moeten voldoende blijven functioneren naarmate het hoeveelheid gebruik toe neemt. In deze context gaat het functioneren dus over het niet schenden van tijdsgrenzen, en de hoeveelheid gebruik over de hoeveelheid data.

De tweede definitie die je noemt (schaalbaarheid voor implementaties die rekbaar zijn op basis van veranderende eisen) omvat in dit geval zowel functionele als niet functionele schaalbaarheid. In mijn definities heb ik het vooral over opschalen, dit is nog een verbeterpunt. Het concept van "veranderende eisen" vind ik een mooie.

Om antwoord te geven op je vraag: Schaalbaarheid is een kwaliteitskenmerk, het daadwerkelijk schalen is een uitoefening van dit kenmerk. Dus, ja, het is mogelijk om schaalbaarheid te toetsen los van daadwerkelijke verandering. Een kwaliteitsanalyse op kenmerken als complexiteit van algoritmes bijvoorbeeld. In de context van functionele schaalbaarheid is dit "to what extent it continues to function properly as the amount of use of the system increases", en van niet-functionele schaalbaarheid "to what extent the quality of that requirement remains acceptable as the use of the system increases".

Ik hoop dat ik hiermee je vraag voldoende heb beantwoord, zo niet hoor ik het graag uiteraard.
\end{minted}

Marlon was tevreden met dit antwoord:
\begin{minted}[bgcolor=codebg,breaklines, breaksymbolleft=]{text}
Duidelijk, dank voor je antwoord, erg interessant.
\end{minted}


%----------------------------------------------------------------------------------------
%	BIBLIOGRAPHY
%----------------------------------------------------------------------------------------

\renewcommand{\bibname}{Literatuurlijst}
\printbibliography[heading=bibintoc]

%----------------------------------------------------------------------------------------

\end{document}  
